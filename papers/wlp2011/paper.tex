\documentclass{llncs}

\usepackage{times}
\usepackage{url}
\def\UrlFont{\tt}
\usepackage{latexsym}
\usepackage{xspace}
\usepackage{hyperref}
\usepackage{color}

\usepackage{listings}
\lstset{aboveskip=1.0ex,
        belowskip=1.0ex,
        showstringspaces=false, % no special string space
        mathescape=true,
        flexiblecolumns=false,
        basewidth=0.52em,
        basicstyle=\small\ttfamily}

\lstset{literate={->}{{$\rightarrow{}\!\!\!$}}3
       }

\lstnewenvironment{curry}{\lstset{backgroundcolor=\color[rgb]{0.9,0.9,0.9}}}{}
\lstnewenvironment{haskell}{}{}
\newcommand{\listline}{\vrule width0pt depth1.75ex}

\newcommand{\code}[1]{\mbox{\small\texttt{#1}}}
\newcommand{\ccode}[1]{``\code{#1}''}
\newcommand{\bs}{\char92\xspace} % backslash
\newcommand{\us}{\char95\xspace} % underscore
\newcommand{\dollar}{\code{\char36}} % dollar in programs
\newcommand{\Choice}[2]{#1 \mathop{?} #2}
\newcommand{\todo}[1]{\fbox{\sc To do: #1}}

\begin{document}
\pagestyle{plain}
\sloppy

\title{Implementing Equational Constraints\\ in a Functional Language}

\author{
Bernd Bra{\ss}el
\kern1em
Michael Hanus
\kern1em
Bj{\"o}rn Peem{\"o}ller
\kern1em
Fabian Reck
}
\institute{
Institut f\"ur Informatik, CAU Kiel, D-24098 Kiel, Germany \\
\email{\{bbr|mh|bjp|fre\}@informatik.uni-kiel.de}
}

\maketitle

\begin{abstract}
KiCS2 is a new system
to compile functional logic programs of the source language Curry
into purely functional Haskell programs.
The implementation is based on the idea to represent
the search space as a data structure and logic variables
as operations that generate their values.
This has the advantage that one can apply various,
in particular, complete search strategies to compute solutions.
However, the generation of all values for logic variables
might be inefficient for applications that
exploit constraints on partial values.
To overcome this drawback, we propose new techniques
to implement equational constraints in this framework.
In particular, we show how unification modulo function evaluation
and functional patterns can be added without sacrificing
the efficiency of the kernel implementation.
\end{abstract}


%%%%%%%%%%%%%%%%%%%%%%%%%%%%%%%%%%%%%%%%%%%%%%%%%%%%%%%%%%%%%%%%%%%%%%%%%%
\section{Introduction}
\label{sec:Introduction}

Functional logic languages combine the most important
features of functional and logic programming in a single language
(see \cite{AntoyHanus10CACM,Hanus07ICLP} for recent surveys).
In particular, they provide higher-order functions and demand-driven
evaluation from functional programming together with logic programming features
like non-deterministic search and computing with partial information
(logic variables).
This combination
has led to new design patterns \cite{AntoyHanus02FLOPS,AntoyHanus11WFLP}
and better abstractions for application programming,
but it put new demands on the implementation side.

Previous implementations of functional logic languages
can be classified into three categories:
(1) designing new abstract machines appropriately supporting
these operational features and implementing them in some (typically,
imperative) language, like C \cite{Lux99FLOPS}
or Java \cite{AntoyHanusLiuTolmach05,HanusSadre99JFLP},
(2) compilation into logic languages (Prolog) and reuse
the existing backtracking implementation for non-deterministic
search as well as logic variables and unification for computing with partial
information \cite{AntoyHanus00FROCOS,Lopez-FraguasSanchez-Hernandez99}, or
(3) compilation into non-strict functional languages like Haskell
and reuse the implementation
of lazy evaluation and higher-order functions
\cite{BrasselHuch07,BrasselHuch09}.
The latter approach requires the implementation
of non-deterministic evaluations but has the advantage
that the explicit handling of non-determinism provides for
various search strategies like depth-first, breadth-first, parallel,
or iterative deepening instead of committing to a fixed (incomplete)
strategy like backtracking \cite{BrasselHuch07}.

In this paper we consider KiCS2 \cite{BrasselFischerHanusReck11},
a new system that compiles functional logic programs of the
source language Curry \cite{Hanus06Curry}
into purely functional Haskell programs.
We have shown in \cite{BrasselFischerHanusReck11}
that this implementation can compete with or outperform
other existing implementations of Curry.
KiCS2 is based on the idea to represent
the search space, i.e., all non-deterministic results of a computation,
as a data structure that can be traversed by operations
implementing various strategies.
Furthermore, logic variables are replaced by generators,
i.e., operations that non-deterministically evaluate to
all possible ground values of the type of the logic variable.
Actually, it has been shown \cite{AntoyHanus06ICLP}
that computing with logic variables by narrowing \cite{Reddy85,Slagle74}
or computing with generators by rewriting
are equivalent, i.e., compute the same results.
Although this implementation technique is correct \cite{Brassel11Thesis},
the generation of all values for logic variables
might be inefficient for applications that
exploit constraints on partial values.
For instance, in Prolog the equality constraint \ccode{X=c(a)}
is solved by instantiating the variable \code{X} to \code{c(a)},
but the equality constraint \ccode{X=Y} is solved by binding \code{X}
to \code{Y} without enumerating any values for \code{X} or \code{Y}.
Therefore, we propose in this paper new techniques
to implement equational constraints in the framework of KiCS2
(note that, in contrast to Prolog, unification is performed
modulo function evaluation in functional logic languages).
Furthermore, we also show how functional patterns \cite{AntoyHanus05LOPSTR},
i.e., patterns containing evaluable operations providing
for more powerful pattern matching than in logic or functional languages,
can be implemented in this framework.
We show that both extensions lead to efficiency improvement
without sacrificing the efficiency of the kernel implementation.

In the next section, we review the source language Curry
and the features that are considered in this paper.
Section~\ref{sec:Compilation} sketches the implementation scheme
of KiCS2.
Sections~\ref{sec:Unification} and~\ref{sec:FuncPatterns}
discuss the extensions to implement unification modulo
functional evaluation and functional patterns, respectively.
Benchmarks demonstrating the usefulness of this scheme
are presented in Section~\ref{sec:Benchmarks}
before we conclude in Section~\ref{sec:Conclusions}.



%%%%%%%%%%%%%%%%%%%%%%%%%%%%%%%%%%%%%%%%%%%%%%%%%%%%%%%%%%%%%%%%%%%%%%%%%%
\section{Curry Programs}
\label{sec:Curry}

The syntax of the functional logic language Curry \cite{Hanus06Curry}
is close to Haskell \cite{PeytonJones03Haskell},
i.e., type variables and names of defined operations usually
start with lowercase letters and the names of type and data constructors
start with an uppercase letter. The application of $f$
to $e$ is denoted by juxtaposition (``$f~e$'').
In addition to Haskell, Curry allows free (logic)
variables in conditions and right-hand sides of defining rules.
Hence, an operation is defined by conditional rewrite rules of the form:
%
\begin{equation}
\label{rule}
f~t_1 \ldots t_n \code{~|~} c \code{~=~} e \code{~~where~} vs \code{~free}
\end{equation}
where the \emph{condition} $c$ is optional and
$vs$ is the list of variables occurring in $c$ or $e$ but not in the
\emph{left-hand side} $f~t_1 \ldots t_n$.

In contrast to functional programming and similarly to logic programming,
operations can be defined by overlapping rules so that
they might yield more than one result on the same input.
Such operations are also called \emph{non-deterministic}.
For instance, Curry offers a \emph{choice} operation that is predefined by
the following rules:
%
\begin{curry}
  x ? _ = x
  _ ? y = y
\end{curry}
%
Thus, we can define a non-deterministic operation \code{aBool} by
\label{ex:aBool}
%
\begin{curry}
  aBool = True ? False
\end{curry}
%
so that the expression \ccode{aBool} has two values:
\code{True} and \code{False}.

If non-deterministic operations are used as arguments in other operations,
a semantical ambiguity might occur. Consider the operations
%
\begin{curry}
  xor True  False = True
  xor True  True  = False
  xor False x     = x

  xorSelf x = xor x x
\end{curry}
%
and the expression \ccode{xorSelf aBool}.
If we interpret this program as a term rewriting system,
we could have the reduction
\begin{haskell}
  xorSelf aBool  $\to~$  xor aBool aBool     $\to~$  xor True aBool
                 $\to~$  xor True False      $\to~$  True
\end{haskell}
leading to the unintended result \code{True}.
Note that this result cannot be obtained if we use a strict strategy
where arguments are evaluated before the function calls.
In order to avoid dependencies on the evaluation strategies
and exclude such unintended results,
Gonz\'alez-Moreno et al.\ \cite{GonzalezEtAl99} proposed
the rewriting logic CRWL as a logical
(execution- and strategy-independent) foundation for declarative
programming with non-strict and non-deterministic operations.  This
logic specifies the \emph{call-time choice} semantics \cite{Hussmann92}
\label{ctc-semantics}
where values of the arguments of an operation are determined before the
operation is evaluated. Note that this does not necessarily require
an eager evaluation of arguments.
Actually, \cite{AlbertHanusHuchOliverVidal05,LopezRodriguezSanchez07}
define lazy evaluation strategies for functional logic programs
with call-time choice semantics where actual arguments passed to
operations are shared. Hence, we can evaluate the expression above
lazily, provided that all occurrences of \code{aBool}
are shared so that all of them reduce either to \code{True} or to \code{False}.
The requirements of this call-time choice semantics are the
reason why it is not simply possible to use list comprehensions
or non-determinism monads for a straightforward implementation
of functional logic programs in Haskell \cite{FischerKiselyovShan09}.

The condition $c$ in rule (\ref{rule}) is typically a
conjunction of \emph{equational constraints}
of the form \code{$e_1$\,=:=\,$e_2$}.
Such a constraint is satisfiable if
both sides $e_1$ and $e_2$ are evaluable to unifiable data terms.
For instance, if the symbol ``\code{++}'' denotes the usual list
concatenation operation, we can define an operation \code{last}
that computes the last element \code{e} of a non-empty list \code{xs}
as follows:
%
\label{ex:last}
\begin{curry}
  last xs | ys++[e]=:=xs = e   where ys,e free  
\end{curry}
%
As in Haskell, the rules defining most functions are
\emph{constructor-based} \cite{ODonnell85}, in (\ref{rule})
$t_1,\ldots,t_n$ are made of variables and/or data constructor symbols
only.  However, Curry also allows \emph{functional pattern}
\cite{AntoyHanus05LOPSTR}, i.e., $t_i$ might contain calls to
defined operations.  For instance, we can define the last element
of a list also by
as:
%
\begin{curry}
  last' (xs++[e]) = e 
\end{curry}
%
Here, the functional pattern \code{(xs++[e])}
states that \code{last'} is reducible
to \code{e} provided that the argument can be matched against
some value of \code{(xs++[e])} where \code{xs} and \code{e} are
free variables.
By instantiating \code{xs} to arbitrary lists, the value
of \code{(xs++[e])} is any list having \code{e} as its last element.
Functional patterns are
a powerful feature to express arbitrary selections in term structures.
For instance, functional patterns supports a straightforward
processing of XML data with incompletely specified
or evolving formats \cite{Hanus11ICLP}.



%%%%%%%%%%%%%%%%%%%%%%%%%%%%%%%%%%%%%%%%%%%%%%%%%%%%%%%%%%%%%%%%%%%%%%%%%%
\section{The Compilation of Scheme of KiCS2}
\label{sec:Compilation}

In this section we sketch the translation of Curry programs
into Haskell programs as performed by KiCS2,
since this is necessary to understand the extensions
described in the subsequent sections.
More details about this translation scheme can be found in
\cite{BrasselFischer08IFL,BrasselFischerHanusReck11}.


As mentioned in the introduction, the KiCS2 implementation
is based on the explicit representation of non-deterministic results
in a data structure.
This is achieved by extending each data type of the source program
by constructors to represent a choice between two values
and a failure, respectively.
For instance, the data type for Boolean values defined in a Curry program by
\begin{curry}
  data Bool = False | True
\end{curry}
is translated into the Haskell data type\footnote{Actually,
our compiler performs some renamings to avoid conflicts with
predefined Haskell entities and introduces type classes
to resolve overloaded symbols like \code{Choice} and \code{Fail}.}
\begin{haskell}
  data Bool = False | True | Choice ID Bool Bool | Fail
\end{haskell}
The first argument of each \code{Choice} constructor of type \code{ID}
is used to implement the call-time choice semantics
discussed in Section~\ref{sec:Curry}.
Since the evaluation of \code{xorSelf aBool} duplicates
the argument operation \code{aBool}, we have to ensure
that both duplicates, which later evaluate to a non-deterministic
choice between two values, are either \code{True} or \code{False}.
This is obtained by assigning a unique identifier (of type \code{ID})
to each \code{Choice}. The difficulty is to get a unique identifier
when needed, i.e., when some operation evaluates to a \code{Choice}.
Since we want to compile into \code{purely} functional programs
(in order to enable powerful program optimizations),
we cannot use unsafe features with side effects to generate
such identifiers.
Hence, we pass a (conceptually infinite) set of identifiers,
also called \emph{identifier supply},
to each operation so that a \code{Choice} can pick an identifier from
this set.
For this purpose, we assume a type \code{IDSupply},
representing an infinite set of identifiers,
with operations
\begin{haskell}
  initSupply  :: IO IDSupply
  thisID      :: IDSupply -> ID
  leftSupply  :: IDSupply -> IDSupply
  rightSupply :: IDSupply -> IDSupply
\end{haskell}
The operation \code{initSupply} creates such a set (at the beginning
of an execution),
the operation \code{thisID} takes some identifier from this set, and
\code{leftSupply} and \code{rightSupply} split this set
into two disjoint subsets without the identifier
obtained by \code{thisID}.
There are different implementations available \cite{AugustssonRittriSynek94}
(see below for a simple implementation) and our implementation
is parametric over concrete implementations of \code{IDSupply}.

When translating Curry to Haskell, KiCS2 adds to each operation
an additional argument of type \code{IDSupply}.
For instance, the operation \code{aBool}
defined in Section~\ref{ex:aBool} is translated into:
\begin{haskell}
  aBool :: IDSupply -> Bool
  aBool s = Choice (thisID s) True False
\end{haskell}
Similarly, the operation
\begin{curry}
  main :: Bool
  main = xorSelf aBool
\end{curry}
is translated into
\begin{haskell}
  main :: IDSupply -> Bool
  main s = xorSelf (aBool (leftSupply s)) (rightSupply s)
\end{haskell}
so that the set \code{s} is split into a set \code{(leftSupply s)}
containing identifier for the evaluation of \code{aBool}
and a set \code{(rightSupply s)} containing identifiers
for the evaluation of operation \code{xorSelf}.

Since all data types are extended by additional constructors,
we must also extend the definition of operations performing
pattern matching. For instance, consider the definition of lists
\begin{curry}
  data List a = Nil | Cons a (List a)
\end{curry}
and an operation to extract the first element of a non-empty list:
\begin{curry}
  head :: List a -> a
  head (Cons x xs) = x
\end{curry}
The type definition is extended as follows:
\begin{haskell}
  data List a = Nil | Cons a (List a) | Choice ID (List a) (List a) | Fail
\end{haskell}
The operation \code{head} is extended by an identifier supply
and further matching rules:
\begin{haskell}
  head :: List a -> IDSupply -> a
  head (Cons x xs)      s = x
  head (Choice i x1 x2) s = Choice i (head x1 s) (head x2 s)
  head _                s = Fail
\end{haskell}
The second rule transforms a non-deterministic argument
into a non-deterministic result and
the final rule returns \code{Fail} in all other cases,
i.e., if \code{head} is
applied to the empty list as well as if the matching argument
is already a failed computation (failure propagation).

To show a concrete example, we use the following implementation
of \code{IDSupply} based on unbounded integers:
%
\begin{haskell}
  type IDSupply = Integer
  initSupply    = return 1
  thisID      n = n
  leftSupply  n = 2*n
  rightSupply n = 2*n+1
\end{haskell}
%
If we apply the same transformation to the rules defining \code{xor}
(actually, \code{xor} performs pattern matching on two arguments
which is avoided in KiCS2 by introducing an intermediate operation,
see \cite{BrasselFischerHanusReck11}) and evaluate the main expression
\code{(main 1)},
we obtain the result
\begin{haskell}
  Choice 2 (Choice 2 False True) (Choice 2 True False)
\end{haskell}
Thus, the result is non-deterministic and contains several choices,
but all of them have the same identifier.
This identifier is used when the values contained in this result
are shown: one has to make \emph{consistent} selections in choices with
same identifiers. Thus, if we select the left branch as the value
of the outermost \code{Choice}, we also have to select the left branch
in the selected argument \code{(Choice 2 False True)} so that only
the value \code{False} is possible here.
Similarly, if we select the right branch as the value of the outermost
\code{Choice}, we also have to select the right branch in
its selected argument \code{(Choice 2 True False)} which yields the only
value \code{False}.
Thus, unintended value \code{True} is not produced.

In order to extract values from a \code{Choice} structure,
we have to traverse it, compute all possible choices
but consider the choice identifiers to make consistent (left/right)
decisions.
The latter requirement can be implemented by storing
the decision already made for some choices during the traversal.
For this purpose, we use the type
\begin{haskell}
  data Decision = NoDecision | ChooseLeft | ChooseRight
\end{haskell}
where \code{NoDecision} represents the fact that the value of a choice
is yet undecided.
Furthermore, we assume operations to lookup the current decision
for a given identifier or change its decision (depending on the implementation
of \code{IDSupply}, KiCS2 supports various implementations
based on memory cells or finite maps):
\begin{haskell}
  lookupChoice :: ID -> IO Decision
  setChoice    :: ID -> Decision -> IO ()
\end{haskell}
%
Now we can print all values contained in a choice structure
in a depth-first manner by the following operation:\footnote{%
Note that this code has been simplified for readability
since the type system of Haskell does not
allow this direct definition.}
\label{sec:printValsDFS}
\begin{haskell}
  printValsDFS :: a -> IO ()
  printValsDFS Fail             = return ()
  printValsDFS (Choice i x1 x2) = lookupChoice i >>= choose
  printValsDFS v                = print v
   where
    choose ChooseLeft  = printValsDFS (try x1)
    choose ChooseRight = printValsDFS (try x2)
    choose NoDecision  = do newDecision ChooseLeft  x1
                            newDecision ChooseRight x2

    newDecision ch x = do setChoice i ch
                          printValsDFS (try x)
                          setChoice i NoDecision
\end{haskell}
This operation prints ignores failures and prints values
that are not a \code{Choice}. For a \code{Choice},
it checks whether a decision for
this identifier has already been made (note that the initial value
for all identifiers is \code{NoDecision}).
If a decision has been made for this choice, it follows this decision.
Otherwise, the left alternative is used and this decision is stored.
After printing
all the values w.r.t.\ this decision,
the decision is undone (like in backtracking)
and the right alternative is used and stored.

In general, this operation is applied to the normal form
of the main expression (where \code{initSupply} is used to
compute an initial identifier supply passed to this expression).
The normal form computation is necessary for structured data like lists
so that a failure or choice in some part of the data is moved to
the root. Other search strategies, like
breadth-first search, iterative deepening, or parallel search,
can be obtained by different implementations of this main operation
to print all values. Instead of printing all values,
one can also collect all values in a tree-like data structure
so that the programmer can implement his own search strategies
(this corresponds to encapsulating search \cite{BrasselHanusHuch04JFLP}).
Finally, one can also have other options rather than printing all values
of the main expression.
For instance, one can easily define operations
to prints only the first solution or one by one upon user requests.
Due to the lazy evaluation strategy of Haskell,
such operations can also be applied to infinite choice structures.

An important optimization, performed by our compiler,
is a \emph{determinism analysis}.
If an operation does not call, neither directly nor indirectly through
other operations, the choice operation \ccode{?},
then it is not necessary to pass an identifier supply
so that the generated code is nearly identical to a corresponding
functional program.
Actually, the benchmarks presented in \cite{BrasselFischerHanusReck11}
show that this implementation outperforms all other Curry implementations
for deterministic operations,
and, for non-deterministic operations, outperforms Prolog-based
implementations of Curry and can compete with MCC \cite{Lux99FLOPS},
a Curry implementation that compiles to C.

As mentioned in the introduction,
KiCS2 translates occurrences of logic variables
into generators. For instance, consider the operation \code{not}
defined by the rules:
\begin{curry}
  not True  = False
  not False = True
\end{curry}
Then the expression \ccode{not x}, where \code{x} is a logic variable,
is translated into \ccode{not aBool} (we omit the \code{IDSupply} arguments
here). The latter expression is evaluated by reducing the
argument \code{aBool} to a choice between \code{True} or \code{False}
followed by applying \code{not} to this choice.
This is similar to a narrowing step on \ccode{not x}
that instantiates the variable \code{x} to \code{True} or \code{False}.
Actually, it has been shown in \cite{AntoyHanus06ICLP}
that narrowing expressions with logic variables
is equivalent to rewriting with generators.
Since such generators are standard
non-deterministic operations, they are translated like any other operation
so that one does not need additional run-time support for them.
However, if equational constraints should be solved,
there are methods which are more efficient than generating all values.
These methods and their implementation are discussed in the
next section.


%%%%%%%%%%%%%%%%%%%%%%%%%%%%%%%%%%%%%%%%%%%%%%%%%%%%%%%%%%%%%%%%%%%%%%%%%%
\section{Equational Constraints and Unification}
\label{sec:Unification}

As known from logic programming, predicates or constraints
are important to restrict the set of intended values
in a non-deterministic computation.
Apart from user-defined predicates,
equational constraints of the form \code{$e_1$\,=:=\,$e_2$}
are the most important kind of constraints.
We have already seen in the operation
\code{last} in Section~\ref{ex:last} a typical application
of an equational constraint.

Due to the presence of non-terminating operations
and infinite data structures,
\ccode{=:=} is interpreted as the \emph{strict equality} on terms
\cite{GiovannettiLeviMoisoPalamidessi91},
i.e., the equation \code{$e_1$\,=:=\,$e_2$} is satisfied iff
$e_1$ and $e_2$ are reducible to the same ground constructor term.
In particular, expressions that do not have a value
are not equal w.r.t.\ \ccode{=:=}, e.g.,
the equational constraint \ccode{head [] =:= head []}
is not satisfiable.\footnote{From now on, we use the
standard notation for lists, i.e., \code{[]} denotes the empty list
and \code{(x:xs)} denotes a list with head element \code{x}
and tail \code{xs}.}

Due to this constructive definition, \ccode{=:=}
can be considered as a binary function defined by the
following rules (here we show only the rules for the Boolean
and list types, where \code{Success} denotes the only constructor
of the type \code{Success} of constraints):
\begin{curry}
  True   =:= True    =  Success
  False  =:= False   =  Success
  []     =:= []      =  Success
  (x:xs) =:= (y:ys)  =  x =:= y & xs =:= ys

  Success & c  =  c
\end{curry}
%
If we translate these operations into Haskell by the scheme
shown in Section~\ref{sec:Compilation},
the following rules are added to these rules
in order to propagate choices and failures:
\begin{haskell}
  Choice i l r =:= y             =  Choice i (l =:= y) (r =:= y)
  x            =:= Choice i l r  =  Choice i (x =:= l) (x =:= r)
  _            =:= _             =  Fail

  Choice i l r & c  =  Choice i (l & c) (r & c)
  _            & _  =  Fail

\end{haskell}
%
Although this is a correct implementation of equational constraints,
it might lead to an unnecessarily large search space
when it is applied to generators representing logic variables.
For instance, consider the following generator for Boolean lists:
\begin{curry}
  aBList = [] ? (aBool : aBList)
\end{curry}
This is translated into Haskell as follows:
\begin{haskell}
  aBList :: IDSupply -> [Bool]
  aBList s = Choice (thisID s) [] (aBool (leftSupply s)
                                   : aBList (rightSupply s))
\end{haskell}
Now consider the equational constraint \ccode{x =:= [True]}.
If the logic variable \code{x} is replaced by \code{aBList},
the translated expression \ccode{aBList s =:= [True]}
creates a search space when evaluating its first argument,
although there is no search required since there is only one possible
binding for \code{x}.
Furthermore and even worse, there is an infinite search space
when unifying two logic variables. For instance,
the expression \ccode{xs=:=ys \& xs++ys=:=[True]}
results in an infinite search space when the logic variables
\code{xs} and \code{ys} are replaced by generators.

To avoid these problems, we have to implement the idea
of the well-known unification principle \cite{Robinson65}.
Instead of enumerating all values for logic variables
occurring in an equational constraint,
we \emph{bind} them to another variable or term.
Since we compile into a purely functional language,
the binding cannot be performed by some side effect.
Instead, we add binding constraints to the computed results.

To implement this idea, we have to distinguish free variables
from ``standard choices'' (corresponding to occurrences of \ccode{?})
in the target code. For this purpose, we refine the definition
of the type \code{ID} as follows:\footnote{For the sake
of simplicity, we consider in the following the
implementation of \code{IDSupply} as unbounded integers.}
\begin{haskell}
  data ID = ID Integer | FreeID Integer
\end{haskell}
The new constructor \code{FreeID} marks a choice corresponding to
a free variable, e.g., the generator for Boolean variables is
redefined as
\begin{haskell}
  aBool s = Choice (FreeID (thisID s)) True False
\end{haskell}
%
If an operation is applied to a free variable and requires its value,
the free variable is transformed into a standard choice.
For this purpose, we define a simple operation to perform
this transformation:
\begin{haskell}
  narrow :: ID -> ID
  narrow (FreeID i) = ID i
  narrow x          = x
\end{haskell}
Furthermore, we use this operation in narrowing steps,
i.e., in all rules operating on \code{Choice} constructors.
For instance, in the implementation of the operation \code{not}
we replace the rule
\begin{haskell}
  not (Choice i x1 x2) s = Choice i (not x1 s) (not x2 s)
\end{haskell}
by the rule
\begin{haskell}
  not (Choice i x1 x2) s = Choice (narrow i) (not x1 s) (not x2 s)
\end{haskell}
With this small change, the run-time system can distinguish
free variables (choices with an identifier of the form \code{(FreeID$\ldots$)})
from other values. This is quite useful at the top-level
to print values containing logic variables in a specific form
rather than enumerating all values, similarly to logic programming systems.
For instance, KiCS2 evaluates the application of \code{head}
to an unknown list as follows:
\begin{haskell}
  Prelude> head xs where xs free
  {xs = (_x2:_x3)} _x2
\end{haskell}
Thus, free variables are prefixed by \code{\us{}x}.

As mentioned above, the consideration of free variables
is relevant in equational constraints where binding constraints are
generated. For this purpose, we introduce a type to represent
a \emph{binding constraints} as pair of a choice identifier
and a decision for this choice:
\begin{haskell}
  data Constraint = ID :=: Decision
\end{haskell}
Furthermore, we extend each data type by the possibility to
add some constraints:
\begin{haskell}
  data Bool   = $\ldots$ | Guard [Constraint] Bool
  data List a = $\ldots$ | Guard [Constraint] (List a)
\end{haskell}
Thus, \code{(Guard $cs$ $v$)} represents a \code{constrained value},
i.e., the value $v$ is only valid if the constraints $cs$ are
consistent. The consistency is checked when a value is extracted
from a choice structure, e.g., by the operation \code{printValsDFS},
whereas the binding constraints are created by
equational constraints: if a free variable
should be bound to a constructor, we create the same
decision as it would be done by the (successful) branch
of the generator. In case of the Boolean values,
this can be implemented by the following additional rules
for \ccode{=:=}:
\begin{haskell}
  Choice (FreeID i) _ _ =:= True   =  Guard [i :=: ChooseLeft ] Success
  Choice (FreeID i) _ _ =:= False  =  Guard [i :=: ChooseRight] Success
\end{haskell}
Hence, the binding of a variable to some known value
is implemented as a binding constraint for the choice identifier
for this variable. However, if we want to bind a variable
to another variable, we cannot store a concrete decision.
Instead, we store the information that the decisions for
both variables, when they are made to extract values,
must be identical. For this purpose we extend the decision
type to cover this information:
\begin{haskell}
  data Decision = $\ldots$ | BindTo ID
\end{haskell}
Furthermore, we add the rule that an equational constraint
between two variables yields a binding for these variables:
\begin{haskell}
  Choice (FreeID i) _ _ =:= Choice (FreeID j) _ _
    =  Guard [i :=: BindTo j] Success
\end{haskell}
%
As mentioned above, the checking of constraints
for consistency is done when values are extracted.
For this purpose, we extend the definition of the corresponding
operations by calling a constraint solver for the constraints.
For instance, the definition of \code{printValsDFS} is extended by
a rule handling constrained values:
%
\begin{haskell}
  $\ldots$
  printValsDFS (Guard constraints x) = solve constraints (printValsDFS x)
  $\ldots$
\end{haskell}
The operation \code{solve} checks the consistency of the constraints
given in the first arguments,
stores further decisions made by the constraints, e.g.,
if variables are bound together,
and evaluates its second argument.
Furthermore, the operations \code{lookupChoice} and \code{setChoice}
are extended to deal with decisions of type \code{BindTo},
i.e., they follow chains of variable bindings.

Due to the extensions described in this section,
we obtain an implementation of a unification procedure
that integrates the demand-driven evaluation of operations
occurring in the arguments.
Benchmarks evaluating the efficiency of this approach are
shown in Section~\ref{sec:Benchmarks}.

\todo{Mention accumulation of constraints in \&?}



%%%%%%%%%%%%%%%%%%%%%%%%%%%%%%%%%%%%%%%%%%%%%%%%%%%%%%%%%%%%%%%%%%%%%%%%%%
\section{Functional Patterns}
\label{sec:FuncPatterns}

A well-known disadvantage of equational constraints
is the fact that \ccode{=:=} is interpreted as strict equality.
Thus, if one uses equational constraints to express
requirements on arguments, the resulting operations might be too strict.
For instance, the equational constraint in the condition
defining \code{last} (see Section~\ref{ex:last})
requires that \code{ys++[e]} as well as \code{xs}
must be reducible to identical values so that the input list \code{xs}
is completely evaluated.
Hence, if \code{failed} denotes an operation whose evaluation fails,
the evaluation of \code{last [failed,True]} has no result.
On the other hand, the evaluation of \code{last' [failed,True]} yields
the value \code{True}, i.e., the definition of \code{last'} is less strict.
This is due to the fact that a functional pattern like
\code{(xs++[e])} abbreviates all values to which it can be evaluated
(by narrowing), like \code{[e]}, \code{[x1,e]}, \code{[x1,x2,e]} etc.
Conceptually, the rule defining \code{last'}
abbreviates the following (infinite) set of rules:
\begin{curry}
  last' [e] = e
  last' [x1,e] = e
  last' [x1,x2,e] = e
  $\ldots$
\end{curry}
As a further example,
consider an operation that returns the first duplicated element in
a list.
Using equational constraint, we can define it as follows:
\begin{curry}
  fstDup xs | xs =:= ys++[e]++zs & elem e ys =:= True & nub ys =:= ys
            = e    where ys,zs,e free
\end{curry}
The first equational constraint is used to split the input list \code{xs}
into three sublists.
The last constraint ensures that the first sublist
\code{ys} does not contain duplicated elements
(the library operation removes all duplicates from a list)
and the second constraint ensures that the first element after \code{ys}
occurs in \code{ys}.
Although this implementation is concise, it cannot be applied
to infinite lists due to the strict interpretation of \ccode{=:=}.
This is not the case if we define this operation by a functional pattern:
\begin{curry}
  fstDup' (ys++[e]++zs) | elem e ys =:= True & nub ys =:= ys
                        = e
\end{curry}
%
Functional patterns are quite useful in situations
where one wants to select parts of some structure
at arbitrary positions. For instance,
functional patterns have been applied to support
the high-level programming of querying or transforming
XML data \cite{Hanus11ICLP}.

Obviously, one cannot implement functional patterns by
a transformation into an infinite set of rules.
Fortunately, functional patterns can be implemented by
a specific \emph{lazy unification} procedure \ccode{=:<=}
\cite{AntoyHanus05LOPSTR}.
For instance, the definition of \code{last'} is transformed into
\begin{curry}
  last' ys | (xs++[e]) =:<= ys  = e   where xs,e free
\end{curry}
The behavior of \ccode{=:<=} is similar to \ccode{=:=}
except for the case that a variable in the left argument
should be bound to some expression: instead of evaluating
the expression to some value and bind the variable to it,
the variable is bound to the \emph{unevaluated} expression
(see \cite{AntoyHanus05LOPSTR} for more details).
Due to this slight change, failures or infinite structures
in actual arguments do not cause problems in the matching
of functional patterns.

This discussion shows the requirements on an implementation
of functional patterns. The general structure
is quite similar to equational constraints
with the exception that variables could be also bound
to unevaluated expressions. If such variables are later
accessed, they are evaluated.
This can be achieved by adding a further alternative
to the type of decisions:
\begin{haskell}
  data Decision = $\ldots$ | LazyBind [Constraint]
\end{haskell}
%
The implementation of the lazy unification operation \ccode{=:<=}
is almost identical to the strict unification operation \ccode{=:=}
as shown in Section~\ref{sec:Unification}.
The only change is in the rules where a free variable occurs
in the left argument. All these rules are replaced by the single
rule
\begin{haskell}
  Choice (FreeID i) _ _ =:<= x
    = Guard [i :=: LazyBind (lazyBind i x)] Success
\end{haskell}
where the auxiliary operation \code{lazyBind}
implements the demand-driven evaluation of the right argument \code{x}:
\begin{haskell}
  lazyBind :: ID -> a -> [Constraint]
  lazyBind i True  = [i :=: ChooseLeft ]
  lazyBind i False = [i :=: ChooseRight]
\end{haskell}
This has the effect that the argument \code{x} is stored
without evaluation (due to the lazy evaluation strategy of the
target language Haskell) as a binding constraint.
However, it is evaluated by \code{lazyBind} when its binding
is required by another part of the computation.
Similarly to equational constraints,
lazy bindings are processed by a solver when values are extracted.
In particular, if a variable has more than one lazy binding constraint
(which is possible if a functional pattern evaluates to a non-linear term),
the corresponding expressions are evaluated and unified
according to the semantics of functional patterns
\cite{AntoyHanus05LOPSTR}.

In order to demonstrate the operational behavior of our implementation,
we sketch the evaluation of the lazy unification constraint
\code{xs++[e]=:<=[failed,True]} that occurs when the expression
\code{last' [failed,True]} is evaluated (here we omit failed branches
and some other details; note that logic variables are replaced
by generators):
\begin{haskell}
  aBList 2 ++ [aBool 3] =:<= [failed,True]
  $\leadsto$ [aBool 4, aBool 3] =:<= [failed, True]
  $\leadsto$ aBool 4 =:<= failed & aBool 3 =:<= True & [] =:<= []
  $\leadsto$ Guard [ 4 :=: LazyBind (lazyBind 4 failed)
           , 3 :=: LazyBind (lazyBind 3 True)] Success
\end{haskell}
If the binding of identifier \code{3} is later required,
\code{(lazyBind 3 True)} is evaluated to \code{[3 :=: ChooseLeft]}
which corresponds to the value \code{True} of the generator \code{aBool}.
Note that the variable with identifier \code{4} does not occur
anywhere else so that the binding \code{(lazyBind 4 failed)}
will never be evaluated, as intended.


         
%%%%%%%%%%%%%%%%%%%%%%%%%%%%%%%%%%%%%%%%%%%%%%%%%%%%%%%%%%%%%%%%%%%%%%%%%%
\section{Benchmarks}
\label{sec:Benchmarks}

Benchmarks for Last with ==, =:= and func. patterns

also short comparison of =:= with PAKCS and MCC


%%%%%%%%%%%%%%%%%%%%%%%%%%%%%%%%%%%%%%%%%%%%%%%%%%%%%%%%%%%%%%%%%%%%%%%%%%
\section{Conclusions and Related Work}
\label{sec:Conclusions}

Future work: adding other constraint systems


\bibliographystyle{plain}
\bibliography{mh}

\end{document}

% LocalWords:  Curry KiCS equational

