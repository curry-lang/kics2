%%%%%%%%%%%%%%%%%%%%%%%%%%%%%%%%%%%%%%%%%%%%%%%%%%%%%%%%%%%%%%%%%%%%%
% This is a paper to benchmark various versions of the KiCS2 system
% and compare its efficiency with other Curry systems (PAKCS, MCC).
%
% This can be executed and formatted by the command
% > ../execpaper -f bench_kics2.tex
%
%%%%%%%%%%%%%%%%%%%%%%%%%%%%%%%%%%%%%%%%%%%%%%%%%%%%%%%%%%%%%%%%%%%%%

\documentclass{scrartcl}

\usepackage{currycode}
\usepackage{graphicx}

\setlength{\parindent}{0ex}

\begin{document}
\sloppy

\title{Benchmarking KiCS2}
\author{Michael Hanus}
\maketitle

\section{Introduction}

This document contains the result of benchmarking various
Curry systems.
%
All benchmarks were executed in the following environment:
\begin{itemize}
\item \runcurry{getOS} machine (named ``\runcurry{getHostName}'')
\item operating system: \runcurry{getSystemID} \runcurry{getSystemRelease}
\item CPU: ``\runcurry{getCPUModel}'' processor containing
      \runcurry{getCoreNumber} cores
\item time limit for each benchmark: \runcurry{return (show timeLimit)} seconds
\item number of iterations for each benchmark:
       \runcurry{return (show numberOfRuns)}
\end{itemize}
     
\section{Benchmarks}

%%% The actual benchmark programs are in a separate Curry module:
\begin{curryprog}
----------------------------------------------------------------------
import Benchmarks
import BenchKiCS2
----------------------------------------------------------------------
\end{curryprog}

The benchmarks are executed with various Curry systems.
The following table summarizes some information about the
actual back ends of these systems.

\begin{center}
\runcurry{distInfosAsTable (PAKCS:kics2Systems++[KiCS2Local []])}
\end{center}

\subsection{Deterministic Benchmarks}

The following diagram compares MCC, PAKCS, and KiCS2
on the classical naive reverse benchmark for different list lengths:

\begin{center}
\runcurry{benchNRevAndPlot 5.2 [500,1000..] "plot_all.pdf"}
\end{center}

Next we test various Curry systems on deterministic operations,
i.e., operations without overlapping rules or free variables:

\begin{center}
\runcurry{mainBench (MCC:PAKCS:kics2Systems) detBench}
\end{center}


\subsection{Non-deterministic DFS Benchmarks}

The following benchmark programs contain non-deterministic operations
or free variables. We evaluate the computation of all
solutions/values via the depth-first strategy (DFS)
with various Curry systems:

\begin{center}
\runcurry{mainBench [PAKCS, MCC, KiCS2 ""] nondetBenchDFS}
\end{center}

These benchmarks contain functional patterns which are not supported in MCC:

\begin{center}
\runcurry{mainBench [PAKCS, KiCS2 ""] funpatBenchDFS}
\end{center}


\subsection{Search Strategy Benchmarks for KiCS2}

The following benchmark programs compare various search strategies
as they are supported in KiCS2.

The first set of benchmarks computes all solutions/values:

\begin{center}
\runcurry{mainBench kics2Systems nondetBench}
\end{center}

The following benchmarks computes the first solution/value
only (since the search space is infinite):

\begin{center}
\runcurry{mainBench kics2Systems nondetBenchFirst}
\end{center}


\subsection{Parallel Non-deterministic Benchmarks}

These benchmarks are executed with different search strategy options
to evaluate the parallel strategy w.r.t.\ the number
of threads used for the execution.

\begin{center}
\runcurry{mainParBench (KiCS2 "") parSearchBench}
\end{center}


\subsection{Encapsulated Search Benchmarks}

\begin{center}
\runcurry{mainBench kics2Systems encapsBench}
\end{center}


\subsection{Binding Benchmarks for KiCS2}

The following benchmarks demonstrate the relevance of the
constraint store of KiCS2. They compare PAKCS with a local installation
of KiCS2 where KiCS2 runs either in standard mode and
in a mode where the constraint store is disabled by the option
\begin{quote}
\texttt{:set ghc -DDISABLE\_CS -fforce-recomp}
\end{quote}

\begin{center}
\runcurry{mainBench [PAKCS, KiCS2Local [], KiCS2Local [WithoutCS]] bindingBench}
\end{center}


\end{document}
