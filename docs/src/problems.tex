\section{Technical Problems}

One can implement distributed systems with \CYS
by the use of the library \code{Network.NamedSocket}
(available in package \code{cpns})
that supports a socket communication with symbolic names
rather than natural numbers.
For instance, this library is the basis of programming
dynamic web pages with the Curry package \code{html}.
However, it might be possible that some technical problems
arise due to the use of named sockets.
Therefore, this section gives some information
about the technical requirements of \CYS and how to solve
problems due to these requirements.

There is one fixed port that is used by the implementation of \CYS:
\begin{description}
\item[Port 8769:] This port is used by the
{\bf Curry Port Name Server} (CPNS) to implement symbolic names for
named sockets in Curry.
If some other process uses this port on the machine,
the distribution facilities defined in the module \code{Network.NamedSocket}
cannot be used.
\end{description}
If these features do not work, you can try to find out
whether this port is in use by the shell command
\ccode{netstat -a | grep 8769} (or similar).

The CPNS is implemented as a demon listening on its port 8769
in order to serve requests about registering a new symbolic
name for a named socket or asking the physical port number
of an registered named socket. The demon will be automatically started for
the first time on a machine when a user runs a program
using named sockets.
It can also be manually started and terminated by the
command \code{curry-cpnsd} (which is available by installing
the package \code{cpns}, e.g., by the command \ccode{cypm install cpnsd})
If the demon is already running,
the command \ccode{curry-cpnsd start}
does nothing (so it can be always executed
before invoking a Curry program using named sockets).

If you detect any further technical problem,
please write to
\begin{center}
\code{kics2@curry-language.org}
\end{center}
