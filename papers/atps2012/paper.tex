\documentclass[english]{lni}

%\IfFileExists{latin1.sty}{\usepackage{latin1}}{\usepackage{isolatin1}}

\usepackage{graphicx}
\usepackage{url}
\usepackage{listings}
\lstset{aboveskip=1.2ex,
        belowskip=1.2ex,
        xleftmargin=2ex,
        showstringspaces=false, % no special string space
        mathescape=true,
        flexiblecolumns=false,
        basewidth=0.52em,
        basicstyle=\small\ttfamily}
\lstset{literate={->}{{$\rightarrow{}\!\!\!$}}3
       }
\lstnewenvironment{curry}{}{}
\newcommand{\listline}{\vrule width0pt depth1.75ex}
\newcommand{\code}[1]{\texttt{\small{}#1}}
\newcommand{\ccode}[1]{``\code{#1}''}
\newcommand{\bs}{\char92} % backslash
\newcommand{\us}{\char95} % underscore

\author{
Michael Hanus
\quad
Bj{\"o}rn Peem{\"o}ller
\quad
Fabian Reck \\
\\
Institut f\"ur Informatik, CAU Kiel, D-24098 Kiel, Germany \\
\texttt{\{mh|bjp|fre\}@informatik.uni-kiel.de}
}
\title{Search Strategies for Functional Logic Programming}
\begin{document}
\maketitle

\begin{abstract}
In this paper we discuss our practical experiences
with the use of different search strategies
in functional logic programs.
In particular, we show that complete strategies,
like breadth-first search or iterative deepening search,
are a viable alternative to incomplete strategies, like depth-first
search, that have been favored in the past for logic programming languages.
\end{abstract}

\section{Introduction}

Functional logic languages combine the most important
features of functional and logic programming in a single language
(see \cite{AntoyHanus10CACM,Hanus07ICLP} for recent surveys).
In particular, they provide higher-order functions and demand-driven
evaluation from functional programming together with logic programming features
like non-deterministic search and computing with partial information
(logic variables).
This combination
led to new design patterns \cite{AntoyHanus02FLOPS,AntoyHanus11WFLP}
and better abstractions for application programming,
e.g., as shown for programming with databases
\cite{BrasselHanusMueller08PADL,Fischer05},
GUI programming \cite{Hanus00PADL},
web programming \cite{Hanus01PADL,Hanus06PPDP,HanusKoschnicke10PADL},
or string parsing \cite{CaballeroLopez99}.
Moreover, it is also a good basis
to teach the ideas of functional and logic programming,
or declarative programming in general,
with a single computation model and programming language
\cite{Hanus97DPLE}.

An important feature of logic programming languages
is non-deterministic search.
In Prolog, which is still the standard language for logic programming,
non-deterministic search is implemented via backtracking,
which corresponds to a depth-first search traversal
through the SLD proof tree \cite{Lloyd87}.
Due to this feature of Prolog,
the idea of logic programming is often reduced to
the combination of unification and backtracking,
as shown by approaches to add logic programming features
to existing functional languages
(e.g., \cite{ClaessenLjungloef00,Hinze01}).
This limited ``backtracking'' view of logic programming
is also harmful to beginners, e.g., when newbies
define their family relationships using a Prolog rule like
\begin{lstlisting}
sibling(X,Y) :- sibling(Y,X).
\end{lstlisting}
In such cases, one has to explain from the beginning
the pitfalls of backtracking which harms the understanding
of declarative programming.
From a declarative point of view,
a logic program defines a set of rules
and a logic programming system tries to find
a solution to a query w.r.t.\ the set of rules.
In order to abstract from operational details,
the search strategy has to be complete.
Due to these considerations,
the functional logic language Curry \cite{Hanus06Curry}
does not fix a particular search strategy
so that different Curry implementations can support
different (or also several) search strategies.
Moreover, Curry implementations also
support encapsulated search
where non-deterministic computations are represented
in a data structure so that different search strategies
can be implemented as tree traversals
\cite{BrasselHanusHuch04JFLP,HanusSteiner98PLILP,Lux99FLOPS}.

In this paper, we present our practical results
with different search strategies implemented
in a new Curry system called
KiCS2 \cite{BrasselHanusPeemoellerReck11}.
KiCS2 compiles Curry programs into Haskell programs
where non-deterministic values or computations are represented
as tree structures so that flexible search strategies
can be supported.
Although the incomplete depth-first search strategy
is the most efficient one (provided that it is able
to find a result value),
we show that complete strategies,
like breadth-first search or iterative deepening search,
are a reasonable alternative that does not force
the programmer to consider the applied search strategy
in his program.

In the next section, we briefly recall some principles
of functional logic programming and the programming language Curry
that are necessary to understand the remaining part of the paper.
The encapsulation of search and the implementation of different
search strategies are discussed in Section~\ref{sec:strategies}.
These strategies are evaluated with a number of benchmarks
in Section~\ref{sec:benchmarks}
before we conclude in Section~\ref{sec:conclusions}.


%%%%%%%%%%%%%%%%%%%%%%%%%%%%%%%%%%%%%%%%%%%%%%%%%%%%%%%%%%%%%%%%%%%%%%%
\section{Functional Logic Programming and Curry}
\label{sec:curry}

Integrated functional logic programming languages
combine features from functional programming and logic programming.
Recent surveys are available in \cite{AntoyHanus10CACM,Hanus07ICLP}.
Curry \cite{Hanus06Curry} is a functional logic language
which extends lazy functional programming as to be found in
Haskell \cite{PeytonJones03Haskell} and additionally supports
logic programming features.
Another functional logic language based on similar principles is
${\cal TOY}$ \cite{Lopez-FraguasSanchez-Hernandez99}.
However, ${\cal TOY}$ does not offer flexible search strategies
by a concept of encapsulating search
(although it provides a concept of nested computation spaces
in order to deal with failures in functional logic
programming \cite{LopezSanchez04,SanchezHernandez06}).
Therefore, we use Curry throughout this paper, although
the concepts presented here could be also integrated in
other functional logic languages.

A Curry program consists of the definition of data types and
operations on these types. 
Since the syntax of Curry is close to Haskell,
variables and function names usually
start with lowercase letters and the names of type and data constructors
start with an uppercase letter. The application of $f$
to $e$ is denoted by juxtaposition (``$f~e$'').
In addition, Curry allows free (logic) 
variables in conditions and right-hand sides of defining rules.
Note that in a functional logic language
operations might yield more than one result on the same input due to 
the logic programming features.
For instance, Curry contains a \emph{choice} operation defined by:
%
\begin{curry}
x ? _ = x
_ ? y = y
\end{curry}
%
The choice operation can be used to define other non-deterministic operations
like
%
\begin{curry}
coin = 0 ? 1  
\end{curry}
Thus, the expression \ccode{coin} has two values: \code{0} and \code{1}.
If expressions have more than one value, one wants to select
intended values according to some constraints,
typically in conditions of program rules.
A \emph{rule} has the form
\begin{curry}
$f~t_1\ldots{}t_n$ | $c$ = $e$
\end{curry}
where the (optional) condition $c$ is a \emph{constraint},
i.e., an expression of the built-in type
\code{Success}. For instance, the trivial constraint
\code{success} is a value of type \code{Success} that
denotes the always satisfiable constraint.
Thus, we say that a constraint $c$ is \emph{satisfied} if it
can be evaluated to \code{success}.
An \emph{equational constraint} $e_1 \,\code{=:=}\, e_2$ is satisfiable
if both sides $e_1$ and $e_2$ are reducible to unifiable values.
%Furthermore, if $c_1$ and $c_2$ are constraints,
%\code{$c_1\,$\&$\,\,c_2$} denotes their concurrent conjunction
%(i.e., both argument constraints are concurrently evaluated).

As a simple example, consider the following Curry program
which defines a polymorphic 
data type for lists and operations to compute the 
concatenation of lists and the last element of a list:\footnote{Note
that lists are a built-in data type with a more convenient syntax, 
e.g., one can write \code{[x,y,z]} instead of \code{x:y:z:[]}
and \code{[a]} instead of the list type \ccode{List a}.}
%
\begin{curry}
data List a = [] | a : List a    -- [a] denotes "List a"

-- "++" is a right-associative infix operator
(++) :: [a] -> [a] -> [a]
[]     ++ ys = ys
(x:xs) ++ ys = x : (xs ++ ys)

last :: [a] -> a
last xs | (ys ++ [z]) =:= xs
        = z                    where ys,z free
\end{curry}
%
Logic programming is supported by admitting function calls with free
variables (e.g., \code{(ys++[z])} in the rule defining \code{last})
and constraints in the condition of a defining rule. In contrast to
Prolog, free variables need to be declared explicitly to make their
scopes clear (e.g., \ccode{where ys,z free} in the example).
A conditional rule is applicable if its condition  is satisfiable.
Thus, the rule defining \code{last} states in its condition
that \code{z} is the last element of a given list
\code{xs} if there exists a list \code{ys}
such that the concatenation of \code{ys} and the one-element list
\code{[z]} is equal to the given list \code{xs}.

Curry also offers standard features of
functional languages, like modules or monadic I/O
which is identical to Haskell's I/O concept \cite{Wadler97}.
Thus, \ccode{IO $\alpha$} denotes the type of an I/O action that returns values
of type $\alpha$.

The operational semantics of Curry is based on an optimal evaluation strategy
\cite{AntoyEchahedHanus00JACM} which is a conservative extension
of lazy functional programming and (concurrent) logic programming.
A big-step and a small-step operational semantics of Curry
can be found in \cite{AlbertHanusHuchOliverVidal05}.
Curry's semantics is sound in the sense of logic programming,
i.e., each computed result is correct and for each
correct result there is a more general computed one
\cite{AntoyEchahedHanus00JACM}.
In order to achieve completeness,
one has to take \emph{all} possible non-deterministic derivation paths
into account. In contrast to Prolog, which fixes
a (potentially incomplete) depth-first search strategy to find solutions,
Curry does not fix a particular search strategy.
Thus, different Curry implementations
can support various search strategies.
For instance, the Curry implementation PAKCS \cite{Hanus10PAKCS},
which compiles Curry programs into Prolog programs,
supports only a depth-first search strategy.
MCC \cite{Lux99FLOPS} compiles Curry programs into C programs
and uses also a depth-first search strategy to find the
solutions of a given top-level goal.
In addition, MCC offers the encapsulation of search
(see below) so that other search strategies,
like a complete breadth-first search strategy, can be used
inside a Curry program.
The Curry implementation KiCS \cite{BrasselHuch07,BrasselHuch09},
which compiles Curry programs into Haskell programs,
offers depth-first and breadth-first search strategies
for top-level goals as well the encapsulation of search
with user-definable strategies.
In this paper we consider the Curry implementation
KiCS2 \cite{BrasselHanusPeemoellerReck11},
which is based on similar ideas than KiCS but uses
a different compilation model avoiding side effects
to enable better optimizations for target programs.
KiCS2 also offers different search strategies for top-level goals
and the encapsulation of search with user-definable strategies,
which is described next.


%%%%%%%%%%%%%%%%%%%%%%%%%%%%%%%%%%%%%%%%%%%%%%%%%%%%%%%%%%%%%%%%%%%%%%%
\section{Search Strategies}
\label{sec:strategies}

As mentioned above, Curry does not enforce a particular search
strategy. A Curry implementation
can provide various search strategies to find solutions
or values for a given constraint or expression.
The most advanced system in this respect is
KiCS2 \cite{BrasselHanusPeemoellerReck11},
which supports the evaluation of top-level expressions
with depth-first search, breadth-first search,
iterative deepening search, or parallel search strategies.
Curry supports this flexibility since all operations with side effects
are collected in monadic I/O operations \cite{Wadler97}.
As a consequence of this computation model,
all non-deterministic computations between I/O operations
must be encapsulated since one can not apply two alternative
I/O operations to an existing ``world''
and non-deterministically proceed with two alternative worlds
(``one can not duplicate the world'').
Therefore, Curry offers the encapsulation of search
by representing non-deterministic computations
in a data structure so that the computation of different solutions
(one solution or all solutions) can conceptually be implemented
as traversals on this data structure.

An early approach
\cite{HanusSteiner98PLILP,Lux99FLOPS}
to encapsulating search in Curry
is based on a primitive search operator
\begin{curry}
try :: (a->Success) -> [a->Success]
\end{curry}
that takes a constraint abstraction, e.g., \code{(\bs{}x->x\,=:=\,coin)},
as input, evaluates it until the first non-deterministic step occurs,
and returns the result: the empty list in case of failure, a list with a single
element in case of success, or a list with at least two elements representing
a nondeterministic choice. For instance,
\code{try\,(\bs{}x->x\,=:=\,coin)} evaluates to
\code{[\bs{}x->x\,=:=\,0,\,\bs{}x->x\,=:=\,1]}.
Based on this primitive, one can define various search strategies
to explore the search space and return its solutions.
\cite{Lux99FLOPS} shows an implementation of this primitive.

Although typical search operators of Prolog,
like \code{findall}, \code{once}, or negation-as-failure,
can be implemented using \code{try},
it became also clear that the combination of
encapsulated search and demand-driven evaluation
and sharing causes further complications \cite{BrasselHanusHuch04JFLP}.
For instance, in an expression like
\begin{curry}
let y=coin in try (\x->x=:=y)
\end{curry}
it is not obvious whether the non-determinism caused by the evaluation
of \code{coin} (introduced outside
but demanded inside the search operator) should be encapsulated or not.
Hence, the result of this expression might depend on the evaluation order.
For instance,
if \code{coin} is evaluated before the \code{try} expression,
it results in two computations where \code{y} is bound to \code{0}
in one computation and to \code{1} in the other computation.
Hence, \code{try} does not encapsulate the nondeterminism of \code{coin}
(this is the semantics of \code{try} implemented in \cite{Lux99FLOPS}).
However, if \code{coin} is evaluated inside the capsule of \code{try}
(because it is not demanded before), then the nondeterminism of
\code{coin} is encapsulated.
These and more peculiarities are discussed in \cite{BrasselHanusHuch04JFLP}.
Furthermore, the order of the solutions might depend on the
textual order of program rules or the evaluation time
(e.g., in parallel implementations).
Therefore, \cite{BrasselHanusHuch04JFLP} contains a proposal
for another primitive search operator:
\begin{curry}
getSearchTree :: a -> IO (SearchTree a)
\end{curry}
It takes an expression and delivers a search tree representing
the search space when evaluating the input:
\begin{curry}
data SearchTree a = Value a | Fail | Or (SearchTree a) (SearchTree a)
\end{curry}
\code{(Value v)} and \code{Fail} represents a single value
or a failure (i.e., no value), respectively,
and \code{(Or t1 t2)} represents a choice (i.e., a non-deterministic value)
between two search trees \code{t1} and \code{t2}.
Since \code{getSearchTree} is an I/O action, its result (in particular,
the order of solutions) depends on the current environment, e.g.,
time of evaluation.
To avoid the complications w.r.t.\ shared variables,
\code{getSearchTree} implements a \emph{strong encapsulation view},
i.e., conceptually, the argument of \code{getSearchTree}
is cloned before the evaluation starts in order to cut any sharing
with the environment.
Furthermore, the structure of the search tree is computed lazily
so that an expression with infinitely many values does not cause
the nontermination of the search operator if one is interested in only
one solution.

This primitive has been implemented for the first time
in KiCS \cite{BrasselHuch07,BrasselHuch09}
and it is also provided in KiCS2 \cite{BrasselHanusPeemoellerReck11}
considered in this paper.
With this primitive,
the programmer can define its own search strategy as
\code{SearchTree} traversals in order to
collect all non-deterministic values into a list structure.
For instance, a depth-first search strategy can be easily defined
as follows:
\begin{curry}
allValuesDFS :: SearchTree a -> [a]
allValuesDFS Fail      = []
allValuesDFS (Value x) = [x]
allValuesDFS (Or x y)  = allValuesDFS x ++ allValuesDFS y
\end{curry}
Note that the lazy evaluation of traversal operations like
\code{allValuesDFS} has the advantage that the search strategy
is decoupled from the control. For instance, we can define
the following operation to print a single value
of a non-deterministic expression:
\begin{curry}
printFirstValueDFS x = getSearchTree x >>= print . head . allValuesDFS
\end{curry}
Thus, \code{(printFirstValueDFS $exp$)} can print some value
even if the non-deterministic expression $exp$ has infinitely many values.
This is in contrast to Prolog's constructs for controlling
search where different operators are necessary to compute
one or all solutions.

It is well known that depth-first search lacks completeness,
i.e., it might not be able to compute all existing values.
For instance, consider the following operation
that non-deterministically returns all increasing numbers
from a given number \code{n}:
\begin{curry}
f n = f (n+1) ? n
\end{curry}
Although \code{0} is a value of \code{(f 0)},
the evaluation of \code{(printFirstValueDFS (f 0))} does not terminate
(provided that the primitive \code{getSearchTree}
explores the non-determinism of \ccode{?} in left-to-right order).
This problem can be avoided with complete search strategies,
like breadth-first search strategy,
which can be defined on \code{SearchTree} structures as follows:
%
\begin{curry}
allValuesBFS :: SearchTree a -> [a]
allValuesBFS t = collect [t]

collect []     = []
collect (t:ts) = values (t:ts) ++ collect (children (t:ts))

values []           = []
values (Fail:ts)    = values ts
values (Value x:ts) = x : values ts
values (Or _ _:ts)  = values ts

children []           = []
children (Fail:ts)    = children ts
children (Value _:ts) = children ts
children (Or x y:ts)  = x:y:children ts
\end{curry}
%
The operation \code{values} extracts the values
in each level of the tree and the operation \code{children}
extracts all direct ancestors of a level in order to recursively
collect their values.

Using \code{allValuesBFS}, we can print a value
of the expression \code{(f 0)} by
%
\begin{curry}
getSearchTree (f 0) >>= print . head . allValuesBFS
\end{curry}
%
In order to abstract from the details of the evaluation order,
it would be preferable to use complete search strategies.
However, complete strategies are often neglected due to performance reasons.
For example, the breadth-first search strategy stores all child nodes 
of a tree level in a list to be explored later, which, as the search space
potentially doubles on each level, may lead to an exponentially
growing memory usage.

Another complete search strategy with a superior memory behavior
is iterative-deepening search.
Basically, it is depth-first search strategy with a depth-bound
which is incremented in each iteration.
Thus, we compute in each iteration a list of values
together with some information whether we have aborted
(due to the depth-bound) the computation of further possible values.
For this purpose, we define list structures that can also end
with an \code{Abort}:
%
\begin{curry}
data AbortList a = Nil | Cons a (AbortList a) | Abort
\end{curry}
%
and define the concatenation on such lists:
%
\begin{curry}
concA :: AbortList a -> AbortList a -> AbortList a
concA Abort       Abort       = Abort
concA Abort       Nil         = Abort
concA Abort       (Cons x xs) = Cons x (concA Abort xs)
concA Nil         ys          = ys
concA (Cons x xs) ys          = Cons x (concA xs ys)
\end{curry}
%
Now we define an operation to collect values in a search tree
within some level bounds (to avoid the repeated collection
of values found in each iteration):
%
\begin{curry}
collectInBounds :: Int -> Int -> SearchTree a -> AbortList a
collectInBounds oldbound newbound st = collectLevel newbound st
 where
  collectLevel _ Fail      = Nil
  collectLevel d (Value x) = if d<=newbound-oldbound then Cons x Nil
                                                     else Nil
  collectLevel d (Or x y)  =
    if d>0 then concA (collectLevel (d-1) x) (collectLevel (d-1) y)
           else Abort
\end{curry}
%
Now, the entire search strategy consists of
repeated calls to \code{collectInBounds} as long as the result list
is aborted. In order to experiment with different parameters,
the initial depth bound and the method to increase the depth bound
in each iteration are passed as parameters to the main operation:
%
\begin{curry}
allValuesIDS :: Int -> (Int -> Int) -> SearchTree a -> [a]
allValuesIDS initdepth incrdepth st =
  iterIDS initdepth (collectInBounds 0 initdepth st)
 where
  iterIDS _ Nil = []
  iterIDS n (Cons x xs) = x : iterIDS n xs
  iterIDS n Abort = let newdepth = incrdepth n
                     in iterIDS newdepth (collectInBounds n newdepth st)
\end{curry}
%
The key advantage of depth-first search
in comparison to breadth-first search is its memory behavior:
whereas breadth-first search has to store all nodes
in a level of the search tree in parallel, depth-first search
only needs to store the nodes in the branch from the root
to the current node under investigation.
Since iterative deepening uses depth-first search in each iteration,
it should have a memory behavior similarly to depth-first search
and, in case of wide search trees,
better than breadth-first search.
The price for this behavior is the recomputation of the initial
goal in each iteration. 
However, since we defined iterative deepening on a search tree,
recomputation is not required. Instead, the
already evaluated part of the search tree is kept in memory.
Therefore, we also implemented the iterative deepening strategy
for top-level goals without an explicit search tree
but with a recomputation of the initial goal in each iteration
(see Section~\ref{sec:benchmarks}).

The various operators to encapsulate search can also
be used to implement an interactive top-level search
to print all values of an expression as requested by the user.
For instance, the following I/O operation interactively 
prints all elements of a given list:
%
\begin{curry}
printResults :: [a] -> IO ()
printResults []     = putStrLn "No more values"
printResults (x:xs) = do print x
                         putStr "More values? "
                         inp <- getLine
                         if inp=="yes" then printResults xs
                                       else done
\end{curry}
%
Hence, an interactive Prolog-like top-level behavior
to show the values of an expression $exp$ in depth-first order
can be obtained by
\begin{curry}
getSearchTree $exp$ >>= printResults . allValuesDFS
\end{curry}
In addition, we can print the results in breadth-first order
by using \code{allValuesBFS} instead of \code{allValuesDFS}.
Based on these ideas, KiCS2 provides an interactive top-level search
where the user can select various search strategies,
e.g., depth-first, breadth-first, iterative deepening, or
an experimental implementation of parallel search.
However, the top-level search in KiCS2 is not implemented
via the primitive encapsulation operators but in
a monadic style (see also \cite{BrasselHanusPeemoellerReck11})
in order to avoid the explicit construction
of the \code{SearchTree} structure.
Thus, in the next section we both compare the different search strategies
introduced above as well as the top-level search
with the encapsulated search, in order to evaluate the potential overhead
caused by abstracting and programming with search structures.


%%%%%%%%%%%%%%%%%%%%%%%%%%%%%%%%%%%%%%%%%%%%%%%%%%%%%%%%%%%%%%%%%%%%%%%%
\section{Benchmarks}
\label{sec:benchmarks}

In this section we evaluate the practical behavior
of the various search strategies discussed so far.
Since they are only supported by the Curry implementation KiCS2,
we use only this system for our evaluation.
A general comparison of KiCS2 and other Curry implementations
can be found in \cite{BrasselHanusPeemoellerReck11}.
Since KiCS2 compiles Curry programs into Haskell programs,
we used the Glasgow Haskell Compiler
(GHC 7.0.4, option -O2) to compile and execute
the generated target programs.
All benchmarks were executed on a 32bit Linux machine
running Ubuntu ???? with an Intel Core 2 Duo (???GHz) processor and 4 GiB RAM.
The timings were performed with the time command measuring the
execution time (in seconds) of a compiled executable for each benchmark.
Due to the lack of precise measurements of the space behavior,
we measured only the execution times where
``oom'' denotes a memory overflow in a computation.

\begin{figure}[ht]
\centering
\begin{tabular}{|@{~~}l@{~~}|*{7}{@{~~}r@{~~}|}}
\hline
Program  &   DFS  & IDS(+1)& IDS(*2)&  eDFS  &  eBFS  &eIDS(+1)&eIDS(*2)\\
\hline
PermSort &  13.12 & 841.58 &  32.77 &  38.48 &  66.82 &  72.64 &  46.79 \\
Last     &        &        &        &        &        &        &        \\
Half     &        &        &        &        &        &        &        \\
Graph    &   1.76 &  52.53 &   2.40 &   3.51 &   4.22 &   4.26 &   3.83 \\
HorseMan &   2.64 & 639.64 &   6.63 &   3.23 &   3.29 &   3.20 &   3.17 \\
MAC      &   6.80 & 544.05 &   7.84 &  21.19 &  18.34 &  18.59 &  21.03 \\
Queens   &  24.74 & 374.64 &  33.62 &  42.71 &  43.98 &  44.01 &  42.65 \\
\hline
\end{tabular}
\caption{Benchmarks: computing all values}
\label{fig:bench-all-solutions}
\end{figure}

Figure~\ref{fig:bench-all-solutions} shows the benchmark results
when all values of a given expression are computed.
All benchmark programs are non-deterministic programs.
``PermSort'' sorts a list containing 15 elements by enumerating
all permutations and selecting the sorted ones,
``Last'' computes the last element \code{x}
of a list \code{xs} containing 100,000 elements
by solving the equation \ccode{ys++[x] =:= xs}
(see Section~\ref{sec:curry}),
``Half'' computes the half \code{y} of a natural number \code{x}
(in Peano representation) by solving the equation \code{y+y=:=x},
``Embed'' searches for an embedding of a message
which is ``hidden'' in another message
(see \cite[Sect.~4.1]{Antoy10JSC}),
\ldots

The columns in Figure~\ref{fig:bench-all-solutions}
are the various search strategies considered in this paper.
``DFS'' denotes the top-level depth-first search strategy
which is similar to \code{allValuesDFS} but implements
this strategy in a monadic style without the explicit
construction of a search tree.
The encapsulated search strategies are prefixed by ``e'',
i.e., ``eDFS'' and ``eBFS'' correspond to \code{allValuesDFS}
and \code{allValuesBFS}, respectively.
``eIDS(+1)'' corresponds to \code{allValuesIDS}
with an initial depth bound of \code{10} and the depth increment
operation \code{(+1)}, whereas ``eIDS(*2)'' uses the same initial depth bound
but the depth increment operation \code{(*2)}, i.e., to depth bound
is doubled in each iteration.
Finally, ``IDS(+1)'' and ``IDS(*2)'' are iterative deepening
strategies with similar parameters but recompute the initial
expression in each iteration in order to trade space for run time,
as discussed in Section~\ref{sec:strategies}.

{\sc should we also show relative run times?}

As one can see, depth-first search is the most efficient
search strategy.
However, complete search strategies are viable alternatives.
Their overhead is acceptable taking into account the fact
that one need not to reason about the details of exploring the
search space. The behavior of iterative deepening is largely
influenced by the parameters.
A constant increment of the depth bound causes a big overhead
compared to doubling the depth bound in each iteration.
Breadth-first search seems also a good strategy
taking into account the size of memory provided by modern computers.
However, there are extreme cases when the search tree is wide
where iterative deepening with recomputation is the single
strategy which is able to compute a result.
This is shown in Figure~\ref{fig:bench-first-solution}
that contains benchmarks to compute only a first value
of an expression.

\begin{figure}[ht]
\centering
\begin{tabular}{|@{~~}l@{~~}|*{7}{@{~~}r@{~~}|}}
\hline
Program  &   DFS  & IDS(+1)& IDS(*2)&  eDFS  &  eBFS  &eIDS(+1)&eIDS(*2)\\
\hline
PermSort &        &        &        &        &        &        &        \\
Last     &        &        &        &        &        &        &        \\
Half     &        &        &        &        &        &        &        \\
\hline
\end{tabular}
\caption{Benchmarks: computing a single value}
\label{fig:bench-first-solution}
\end{figure}


%%%%%%%%%%%%%%%%%%%%%%%%%%%%%%%%%%%%%%%%%%%%%%%%%%%%%%%%%%%%%%%%%%%%%%%
\section{Conclusions}
\label{sec:conclusions}

We discussed the use of different search strategies
in functional logic programs.
In order to enable user-programmable search strategies,
modern functional logic languages like Curry
provide primitives to represent non-deterministic computations
or values as data structures that can be traversed like
any term structure.
The Curry implementation KiCS2, considered in this paper,
provides a primitive \code{getSearchTree}
that returns a tree representation of a non-deterministic computation.
This representation can be used to define various search strategies,
like depth-first search, breadth-first search, or iterative deepening
search.

We have practically evaluated and compared
the efficiency of these strategies.
The benchmarks indicated that complete strategies
are a viable alternative to incomplete strategies,
that have been favored in the past due to limited memory requirements.
Thus, it could be reasonable to make
complete strategies the default in functional logic languages.
This would have a good impact on teaching declarative programming
to beginners.
Moreover, decoupling non-deterministic programs
from their search strategy could lead to a more declarative
programming style (instead of the use of predicates with
side effects as in Prolog) and enable more potential for
optimization, e.g., parallel search strategies.
Furthermore, if efficiency and memory limitations are important,
one can still use an efficient strategy like depth-first search
provided that it is able to find all solutions
(e.g., in case of a finite search space).
It is an interesting topic for future work
to statically approximate situations where
the use of incomplete strategies is sufficient.


%\bibliography{mh}
\bibliography{paper}

\end{document}
