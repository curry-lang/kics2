\section{Overview of \CYS}

\subsection{Installation}
\label{sec-install}

This version of \CYS has been developed and
tested on Linux systems.
In principle, it should be also executable on other
platforms on which a Haskell implementation (Glasgow Haskell Compiler
and Cabal) exists, like in many Linux distributions, Sun Solaris,
or Mac OS X systems.

Installation instructions for \CYS can be found in
the file \code{INSTALL.txt} stored in the \CYS installation directory.
Note that there are two possibilities to install \CYS:
\begin{description}
\item[Global installation:]\index{global installation}\index{installation!global}
\CYS is installed in some global system directory where users
have no write permission. In this case, some options for experimenting
with \CYS (like \code{supply} or \code{ghc}, see below)
are not available (since they require the recompilation of parts of
the installed system).
\item[Local installation:]\index{local installation}\index{installation!local}
\CYS is installed in some local user directory where the user
has write permission and the option \code{GLOBALINSTALL}
in the \code{Makefile} of the \CYS installation is set as follows:
\begin{lstlisting}
GLOBALINSTALL=no
\end{lstlisting}
In this case, all options of \CYS are available.
\end{description}
%
Furthermore, \CYS can be installed with experimental support
for profiling\index{profiling}
of executables. To use profiling, two requirements have to be met:
\begin{itemize}
\item The libraries that are shipped with the GHC that is used by \CYS
have to be installed with profiling enabled. This is the default for the system
libraries contained in the GHC release, but may not be the case for additional
libraries.
\item The \code{Makefile} of \CYS contains an option \code{PROFILING}
which has to be set to \code{yes} to enable profiling support. You may
either change the Makefile to
\begin{lstlisting}
PROFILING = yes
\end{lstlisting}
or specify this setting while starting the installation process using
\begin{lstlisting}
make <optional target> PROFILING=yes
\end{lstlisting}
\end{itemize}

In the following, \cyshome denotes the installation directory
of the \CYS installation.


\subsection{General Use}
\label{sec-general}

All executables required to use the different components
of \CYS are stored in the directory \code{\cyshome/bin}.
You should add this directory
to your path (e.g., by the \code{bash} command
\ccode{export PATH=\cyshome/bin:\$PATH}).

The source code of the Curry program
must be stored in a file with the suffix \ccode{.curry},
e.g., \code{prog.curry}.
Literate programs must be stored in files with the extension \ccode{.lcurry}.

Since the translation of Curry programs with \CYS creates
some auxiliary files (see Section~\ref{sec-auxfiles} for details),
you need write permission
in the directory where you have stored your Curry programs.
%Moreover, the current implementation also recompiles
%system libraries according to the setting of some options.
%Therefore, the \CYS system should be locally installed
%in your user account.
The auxiliary files for all Curry programs in the current
directory can be deleted by the command\pindex{cleancurry}
\begin{curry}
cleancurry
\end{curry}
(this is a shell script stored in the \code{bin} directory of the
\CYS installation, see above).
The command
\begin{curry}
cleancurry -r
\end{curry}
also deletes the auxiliary files in all subdirectories.



\subsection{Restrictions}
\label{sec-restrictions}

There are a few minor restrictions on Curry programs
when they are processed with \CYS:
\begin{itemize}
\item
\index{singleton variables}\index{variables!singleton}
\emph{Singleton pattern variables}, i.e., variables that occur only once
in a rule, should be denoted as an anonymous variable \ccode{_},
otherwise the parser will print a warning since this is a
typical source of programming errors.
\item
\CYS translates all \emph{local declarations} into global functions with
additional arguments (``lambda lifting'', see Appendix~D of the
Curry language report).
Thus, in the various run-time systems, the definition of
functions with local declarations look different from
their original definition (in order to see the result
of this transformation, you can use the \cb, see
Section~\ref{sec-currybrowser}).
\item \index{tabulator stops}
Tabulator stops instead of blank spaces in source files are
interpreted as stops at columns 9, 17, 25, 33, and so on.
In general, tabulator stops should be avoided in source programs.
\item
Encapsulated search\index{encapsulated search}:
The general definition of encapsulated search of the Curry report
\cite{HanusSteiner98PLILP} is not supported.
Thus, the corresponding prelude operations like
\code{try}\pindex{try},
\code{solveAll}\pindex{solveAll},
\code{once}\pindex{once},
\code{findall}\pindex{findall}, or
\code{best}\pindex{best}
are not defined in the \CYS prelude.
However, \CYS supports appropriate alternatives
to encapsulate non-deterministic computations:
\begin{description}
\item[Strong encapsulation:]
This means that all potential
non-determinism is encapsulated. Since this might
result in dependencies on the evaluation strategy
(see \cite{BrasselHanusHuch04JFLP} for a detailed discussion),
this kind of encapsulation is only available as I/O operations.
For instance, the library \code{AllSolutions}\pindex{AllSolutions}
(Section~\ref{Library:AllSolutions})
defines the operation
\begin{curry}
getAllValues :: a -> IO [a]
\end{curry}
to compute all values of a given argument expression.
There is also the library \code{SearchTree}\pindex{SearchTree}
(Section~\ref{Library:SearchTree})
which supports user-programmable search strategies
and contains some predefined strategies
like depth-first, breadth-first, iterative deepening search.
\item[Weak encapsulation:]
This means that only the non-determinism
defined inside an encapsulation operator is encapsulated.
Conceptually, these operators are offered as
\emph{set functions}\index{set functions} \cite{AntoyHanus09}
which compute the set of all results but do not encapsulate
non-determinism in the actual arguments.
See the library \code{SetFunctions}
(Section~\ref{Library:SetFunctions}) for more details.
\end{description}
\item
Concurrent computations\index{concurrency}
based on the suspension of expressions
containing free variables are not yet supported.
\CYS supports \emph{value generators} for free variables
so that a free variable is instantiated when its value is demanded.
For instance, the initial expression
\begin{curry}
x == True where x free
\end{curry}
is non-deterministically evaluated to \code{False} and \code{True}
by instantiating \code{x} to \code{False} and \code{True}, respectively.
Thus, a computation is never suspended due to free variables.
This behavior also applies to free variables of primitive types
like integers. For instance, the initial expression
\begin{curry}
x*y=:=1 where x,y free
\end{curry}
is non-deterministically evaluated to the two solutions
\begin{curry}
{x = -1, y = -1} True
{x = 1, y = 1} True
\end{curry}
\item
Unification is performed without an occur check.
\item
There is currently no general connection to external constraint solvers.
\end{itemize}


\subsection{Modules in \CYS}
\label{sec-modules}

\CYS searches for imported modules in various directories.
By default, imported modules are searched in the directory
of the main program and the system module directory
\ccode{\cyshome/lib}.
This search path can be extended
by setting the environment variable \code{CURRYPATH}\pindex{CURRYPATH}
(which can be also set in a \CYS session by the option
\ccode{:set path}\pindex{path}\pindex{:set path},
see below)
to a list of directory names separated by colons (\ccode{:}).
In addition, a local standard search path
can be defined in the \ccode{.kics2rc} file
(see Section~\ref{sec-customization}).
Thus, modules to be loaded are searched in the following
directories (in this order, i.e., the first occurrence of a module file
in this search path is imported):
\begin{enumerate}
\item Current working directory (\ccode{.}) or directory prefix
of the main module (e.g., directory \ccode{/home/joe/curryprogs}
if one loads the Curry program \ccode{/home/joe/curryprogs/main}).
\item The directories enumerated in the environment variable \code{CURRYPATH}.
\item The directories enumerated in the \ccode{.kics2rc} variable
      \ccode{libraries}.
\item The directory \ccode{\cyshome/lib}.
\end{enumerate}
The same strategy also applies to modules with a hierarchical
module name with the only difference that the hierarchy prefix
of a module name corresponds to a directory prefix of the module.
For instance, if the main module is stored in directory \code{MAINDIR}
and imports the module \code{Test.Func}, then the module stored in
\code{MAINDIR/Test/Func.curry} is imported (without setting
any additional import path) according to the
module search strategy described above.

Note that the standard prelude (\code{\cyshome/lib/Prelude.curry})
will be always implicitly imported to all modules if a module
does not contain an explicit import declaration for the module
\code{Prelude}.

%%% Local Variables:
%%% mode: latex
%%% TeX-master: "manual"
%%% End:
