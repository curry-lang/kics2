\documentclass[utf8]{beamer}

\usepackage{graphicx}
\usepackage{color}
\usepackage{listings}
\usetheme{Boadilla}

\setbeamertemplate{footline}%{infolines theme}
{
\leavevmode%
\hbox{%
\begin{beamercolorbox}[wd=.333333\paperwidth,ht=2.25ex,dp=1ex,center]{author in head/foot}%
\usebeamerfont{author in head/foot}\insertshortauthor%~~(\insertshortinstitute)
\end{beamercolorbox}%
\begin{beamercolorbox}[wd=.333333\paperwidth,ht=2.25ex,dp=1ex,center]{title in head/foot}%
\usebeamerfont{title in head/foot}\insertshorttitle
\end{beamercolorbox}%
\begin{beamercolorbox}[wd=.333333\paperwidth,ht=2.25ex,dp=1ex,right]{date in head/foot}%
\usebeamerfont{date in head/foot}\insertshortdate{}\hspace*{2em}
\insertframenumber{} / \inserttotalframenumber\hspace*{2ex}
\end{beamercolorbox}}%
\vskip0pt%
}

\newenvironment{program}{\begin{semiverbatim}\small}{\end{semiverbatim}}

\begin{document}


\title[KiCS2]{Implementation of Advanced Features in KiCS2}

\date{June 22, 2011}

\author[B. Braßel, M. Hanus, B. Peemöller,  \underline{F. Reck}]
 {Bernd Braßel \and Michael Hanus \\
   Björn Peemöller \and \underline{Fabian Reck}\\
  \texttt{\{bbr, mh, bjp, fre\}@informatik.uni-kiel.de}}

\institute{Kiel University}

\begin{frame}
\titlepage
\end{frame}

\begin{frame}[fragile]
\frametitle{Previously on KiCS2}
\begin{itemize}
\item KiCS2 is a compiler from Curry to Haskell
\item Non-determinism is explicitly represented in
      the data structures
\item Unique identifiers are used to obtain
      call-time choice semantics
\item Deterministic and non-deterministc versions
      of higher order functions are generated to
      improve performance
\end{itemize}

\begin{block}{Haskell}
\begin{program}
data Bool = False | True | Choice ID Bool Bool | Fail
\end{program}
\end{block}
\end{frame}

\begin{frame}[fragile]
\frametitle{Search}

\begin{itemize}
 \item Different search strategies traverse the search tree
 \item Decisions for choices have to be managed
 \item Strategies: depth-first-search, breadth-first-search,
       iterative deepening, parallel search
\end{itemize}

\begin{block}{Haskell}
\begin{program}
data Choice = ChooseLeft
            | ChooseRight
            | NoChoice

data Try a = Val a
           | Fail
           | Choice ID a a

try :: a -> Try a
\end{program}
\end{block}

\end{frame}

\begin{frame}[fragile]
\frametitle{Depth-First-Search}

\begin{block}{Haskell}
\begin{program}
printAll :: a -> IO ()
printAll a = dfs (try (normalForm a))

dfs :: Try a -> IO ()
dfs (Val a)        = print a
dfs Fail           = return ()
dfs (Choice i l r) = lookupChoice i >>= choose
 where choose ChooseLeft  = dfs (try l)
       choose ChooseRight = dfs (try r)
       choose NoChoice    = do setChoice i ChooseLeft
                               dfs (try l)
                               setChoice i ChooseRight
                               dfs (try r)
                               setChoice i NoChoice
\end{program}
\end{block}

%If no choice has been made for the current identifyer, both alternatives
%are taken. The choice has to be undone in the end.

\end{frame}

\begin{frame}[fragile]
\frametitle{Free Variables as Generators}

Idea: Free Variables non-deterministically generate every value
of the variables type:

\begin{block}{Curry}
\begin{program}
aBool :: Bool
aBool = x where x free

aList :: [Bool]
aList = xs where xs free 
\end{program}
\end{block}

\begin{block}{Haskell}
\begin{program}
aBool :: ID -> Bool
aBool i = Choice i True False

aList :: ID -> [Bool]
aList i = Choice i [] (aBool (left i) : aList (right i))
\end{program}
\end{block}
\end{frame}

\begin{frame}[fragile]
\frametitle{Unification}
\begin{block}{Haskell}
\begin{program}
(=:=) :: a -> a -> Success

[]           =:= []           = Success
(x:xs)       =:= (y:ys)       = x =:= y \& xs =:= ys
Choice i l r =:= ys           = Choice i (l =:= ys) (r =:= ys)
xs           =:= Choice i l r = Choice i (xs =:= l) (xs =:= r)
_            =:= _            = Fail

Success      \& s = s
Choice i l r \& s = Choice i (l \& s) (r \& s)
_            \& _ = Fail
\end{program}
\end{block}
\begin{itemize}
 \item  \verb!aList 1 =:= [True]! unnecessarily creates search space
 \item Even infinite search space when unifying two variables
\end{itemize}
\end{frame}

\begin{frame}[fragile]
\frametitle{Improved Unification}

\begin{itemize}
\item Distinguish free variables from normal choices  
\item Represent necessary decisions in data type as constraint
\end{itemize}

\begin{block}{Representation of free variables}
\begin{program}
data ID = ID Integer | FreeID Integer

aList i = Choice (free i) [] (aBool (left i) : aList (right i))
\end{program}
\end{block}

\begin{block}{Representation of search constraints}
\begin{program}
data [a] = ... | Guard [Constraint] [a]

data Constraint = ID :=: Choice
\end{program}
\end{block}
\end{frame}

\begin{frame}[fragile]
\frametitle{Example of Improved Unification}
\begin{block}{Unification with variables}
\begin{program}
Choice (FreeID i) _ _ =:= [True] 
-> Guard [ i       :=: ChooseRight      
         , left i  :=: ChooseRight
         , right i :=: ChooseLeft
         ] Success
\end{program}
\end{block}
\pause
\begin{block}{Unification of two variables}
\begin{program}
data Choice = ... | BindTo ID

Choice (FreeID i) _ _ =:= Choice (FreeID j) _ _
-> Guard [i :=: BindTo j] Success
\end{program}
\end{block}
\begin{itemize}
\item Information that a variable is bound to another is stored
\item Search has to be modified
\end{itemize}
\end{frame}



\begin{frame}[fragile]

\frametitle{Search}

\begin{block}{Modification}
\begin{program}

data Try a = ... | Guard [Constraint] a | Free ID a a

dfs (Guard constraints x) = solve constraints (dfs (try x))

dfs (Free i l r) = lookupChoice i >>= choose
   where choose ChooseLeft  = dfs (try l)
         choose ChooseRight = dfs (try r)
         choose NoChoice    = printFreeVar i
\end{program}
\end{block}

\begin{itemize}
\item \verb!BindTo! is handled transparently by
      \verb!lookupChoice! and \verb!setChoice!
\item \verb!solve! checks if the constraints are satisfiable,
      stores the decisions, continues the search and resets the choices
\end{itemize}
\end{frame}

\begin{frame}[fragile]
\frametitle{Function Patterns}
\begin{block}{Curry}
\begin{program}
last (xs ++ [x]) = x
\end{program}
\end{block}
\begin{block}{Desugared Curry}
\begin{program}
last ys = (xs ++ [x]) =:= ys &> x where x,xs free
\pause
> last [error, True]
error
\pause
last ys = (xs ++ [x]) =:<= ys  &> x where x,xs free 

> last [error,True]
True
\end{program}
\end{block}

With \verb!(=:<=)! variables are bound to expressions
rather than values.
\end{frame}

\begin{frame}[fragile]
\frametitle{Lazy Unification}

{\bf Idea:} Delay the unification until the variable is used

\begin{block}{Haskell}
\begin{program}
data Choice = ChooseLeft | .. | LazyBind [Constraint]

(=:<=) :: a -> a -> Success
Choice (FreeID i) _ _ =:<= x 
  = Guard [i :=: LazyBind (lazyBind i x)] Success 

lazyBind :: ID -> a -> [Constraint]
lazyBind i True  = [i :=: ChooseRight]
lazyBind i False = [i :=: ChooseLeft ] 
\end{program}
\end{block}

\end{frame}

\begin{frame}[fragile]
\frametitle{Lazy Unification}
\begin{block}{Example}
\begin{program}

(&>) :: Success -> a -> a

>  aList 2 ++ [aBool 3] =:<= [error,True] &> aBool 3

-- Branches failing due to unification omitted

-> [aBool 4, aBool 3] =:<= [error, True] &> aBool 3

-> aBool 4 =:<= error & aBool 3 =:<= True & [] =:<= [] &> aBool 3

-> Guard [ 4 :=: LazyBind (lazyBind 4 error)
         , 3 :=: LazyBind (lazyBind 3 True)] Success &> aBool 3

-> aBool 3 

-> True
\end{program}
\end{block}

\end{frame}

\begin{frame}[fragile]
\frametitle{Higher Order}

For Higher Order we use the Haskell features:

\begin{block}{Simple Higher Order}
\begin{program}
-- Curry
map :: (a -> b) -> [a] -> [b]

-- Haskell
d_map   :: (a -> b) -> [a] -> [b]
nd_map  :: (a -> ID -> b) -> [a] -> ID -> [b]
\end{program}
\end{block}

{\bf Problem:} \verb!map (id ? not) [True,False]!


\pause

\begin{block}{Non-Deterministic Higher Order}
\begin{program}
data Func a b = Func (a -> ID -> b)
              | Choice ... | Guard  ... | Fail 

nd_map :: Func a b -> [a] -> ID -> [b]
\end{program}
\end{block}
\end{frame}


\begin{frame}[fragile]
\frametitle{Higher Order Datastructures}
\begin{block}{Curry}
\begin{program}
data Map k v = Map (k -> k -> Bool) ...

empty :: (k -> k -> Bool) -> Map k v
\end{program}
\end{block}

\begin{block}{Haskell}
\begin{program}
d_empty  :: (k -> k -> Bool) -> Map k v
nd_empty :: Func k (Func k Bool) -> ID -> Map k v
\pause
data Map k v = D_Map  (k -> k -> Bool) ...
             | ND_Map (Func k (Func k Bool)) ...
             | Choice ...
\end{program}
\end{block}
\begin{itemize}
\item Use deterministic constructor in deterministic context
      and non-deterministic constructor in non-deterministic context
\item Consider functions that use data types with higher order constructors
      as higher order functions
\end{itemize}
\end{frame}
\end{document}