\section{Extensions}
\label{sec-extensions}

\CYS supports some extensions in Curry programs that are not (yet)
part of the definition of Curry. These extensions are described below.

\subsection{Recursive Variable Bindings}

Local variable declarations (introduced by \code{let}\pindex{let}
or \code{where}\pindex{where}) can be (mutually) recursive in \CYS.
For instance, the declaration
\begin{curry}
ones5 = let ones = 1 : ones
         in take 5 ones
\end{curry}
introduces the local variable \code{ones} which is bound
to a \emph{cyclic structure}\index{cyclic structure}
representing an infinite list of \code{1}'s.
Similarly, the definition
\begin{curry}
onetwo n = take n one2
 where
   one2 = 1 : two1
   two1 = 2 : one2
\end{curry}
introduces a local variables \code{one2} that represents
an infinite list of alternating \code{1}'s and \code{2}'s
so that the expression \code{(onetwo 6)} evaluates to \code{[1,2,1,2,1,2]}.


\subsection{Functional Patterns}

Functional patterns \cite{AntoyHanus05LOPSTR} are a useful extension
to code operations in a more readable way. Furthermore,
defining operations with functional patterns avoids problems
caused by strict equality (\ccode{=:=}) and leads to programs
that are potentially more efficient.

Consider the definition of an operation to compute the last element
of a list \code{xs} based on the prelude operation \ccode{++}
for list concatenation:
\begin{curry}
last xs | _++[y] =:= xs  = y   where y free
\end{curry}
Since the equality constraint \ccode{=:=} evaluates both sides
to a constructor term, all elements of the list \code{xs} are
fully evaluated in order to satisfy the constraint.

Functional patterns can help to improve this computational behavior.
A \emph{functional pattern}\index{functional pattern}\index{pattern!functional}
is a function call at a pattern position. With functional patterns,
we can define the operation \code{last} as follows:
\begin{curry}
last (_++[y]) = y
\end{curry}
This definition is not only more compact but also avoids the complete
evaluation of the list elements: since a functional pattern is considered
as an abbreviation for the set of constructor terms obtained by all
evaluations of the functional pattern to normal form (see
\cite{AntoyHanus05LOPSTR} for an exact definition), the previous
definition is conceptually equivalent to the set of rules
\begin{curry}
last [y] = y
last [_,y] = y
last [_,_,y] = y
$\ldots$
\end{curry}
which shows that the evaluation of the list elements is not demanded
by the functional pattern.

In general, a pattern of the form \code{($f$ $t_1$\ldots$t_n$)} ($n>0$)
is interpreted as a functional pattern if $f$ is not a visible constructor
but a defined function that is visible in the scope of the pattern.

It is also possible to combine functional patterns with
as-patterns.\index{as-pattern}\pindex{"@}
Similarly to the meaning of as-patterns
in standard constructor patterns,
as-patterns in functional patterns are interpreted
as a sequence of pattern matching where the variable of the as-pattern
is matched before the given pattern is matched.
This process can be described by introducing an auxiliary operation
for this two-level pattern matching process.
For instance, the definition
\begin{curry}
f (_ ++ x@[(42,_)] ++ _) = x
\end{curry}
is considered as syntactic sugar for the expanded definition
\begin{curry}
f (_ ++ x ++ _) = f' x
 where
  f' [(42,_)] = x
\end{curry}
However, as-patterns are usually implemented
in a more efficient way without introducing auxiliary operations.

\subsection{Order of Pattern Matching}

Curry allows multiple occurrences of pattern variables
in standard patterns. These are an abbreviation of equational constraints
between pattern variables.
Functional patterns might also contain multiple occurrences of
pattern variables.
For instance, the operation
\begin{curry}
f (_++[x]++_++[x]++_) = x
\end{curry}
returns all elements with at least two occurrences in a list.

If functional patterns as well as multiple occurrences of
pattern variables occur in a pattern defining an operation,
there are various orders to match an expression against such
an operation. In the current implementation, the order
is as follows:
\begin{enumerate}
\item Standard pattern matching: First, it is checked whether
the constructor patterns match. Thus, functional patterns
and multiple occurrences of pattern variables are ignored.
\item Functional pattern matching: In the next phase,
functional patterns are matched but occurrences of standard
pattern variables in the functional patterns are ignored.
\item Non-linear patterns: If standard and functional pattern matching
is successful, the equational constraints which correspond
to multiple occurrences pattern variables are solved.
\item Guards: Finally, the guards supplied by the programmer
are checked.
\end{enumerate}
The order of pattern matching should not influence the computed
result. However, it might have some influence on the termination
behavior of programs, i.e., a program might not terminate
instead of finitely failing.
In such cases, it could be necessary to consider the influence
of the order of pattern matching. Note that other orders of pattern matching
can be obtained using auxiliary operations.



\subsection {Records}
\label{records}

A record is a data structure for bundling several data of various types.
It consists of typed data fields where each field is associated with
a unique label. These labels can be used to construct, select or update
fields in a record.


Unlike labeled data fields in Haskell, records are
not syntactic sugar but a real extension of the
language\footnote{The current version allows to transform records
  into abstract data types. Future extensions may not have
  this facility.}.
The basic concept is described in \cite{Leijen05} but the current
version does not yet provide all features mentioned there.
The restrictions are explained in Section~\ref{sec-restrinrecs}.

\subsubsection{Record Type Declaration}
\label{sec-recordtypedecl}

It is necessary to declare a record type before a record
can be constructed or used. The declaration has the following form:
\begin{curry}
type $R$ $\alpha_1$ $\ldots$ $\alpha_n$ = { $l_1$ :: $\tau_1$,$\ldots$, $l_m$ :: $\tau_m$ }
\end{curry}
It introduces a new $n$-ary record type $R$ which represents a
record consisting of $m$ fields. Each field has a unique label $l_i$
representing a value of the type $\tau_i$. Labels
are identifiers which refer to the corresponding
fields. The following examples define some record types:
\begin{curry}
type Person = {name :: String, age :: Int}
type Address = {person :: Person, street :: String, city :: String}
type Branch a b = {left :: a, right :: b}
\end{curry}
It is possible to summarize different labels which have the same
type. For instance, the record \code{Address} can also be declared as follows:
\begin{curry}
type Address = {person :: Person, street,city :: String}
\end{curry}
The fields can occur in an arbitrary order. The example above
can also be written as
\begin{curry}
type Address = {street,city :: String, person :: Person}
\end{curry}
The record type can be used in every type expression to represent
the corresponding record, e.g.
\begin{curry}
data BiTree = Node (Branch BiTree BiTree) | Leaf Int
\end{curry}
\begin{curry}
getName :: Person -> String
getName $\ldots$
\end{curry}

Labels can only be used in the context of
records. They do not share the name space with
functions/constructors/variables or type constructors/type variables.
For instance it is possible to use
the same identifier for a label and a function at the same time. Label
identifiers cannot be shadowed by other identifiers.

Like in type synonym declarations, recursive or mutually
dependent record declarations are not allowed. Records can only
be declared at the top level. Further restrictions are described in
section \ref{sec-restrinrecs}.


\subsubsection{Record Construction}
\label{sec-recordconstr}

The record construction generates a record with initial values for
each data field. It has the following form:
\begin{curry}
{ $l_1$ := $v_1$,$\ldots$, $l_m$ := $v_m$ }
\end{curry}
It generates a record where each label $l_i$ refers to the
value $v_i$. The type of the record results from the record type
declaration where the labels $l_i$ are defined.
A mix of labels from different
record types is not allowed. All labels must be specified with
exactly one assignment. Examples for record constructions are
\begin{curry}
{name := "Johnson", age := 30}     -- generates a record of type 'Person'
{left := True, right := 20}        -- generates a record of type 'Branch'
\end{curry}
Assignments to labels can occur in an arbitrary order. For instance a
record of type \code{Person} can also be generated as follows:
\begin{curry}
{age := 30, name := "Johnson"}     -- generates a record of type 'Person'
\end{curry}
Unlike labeled fields in record type declarations,
record constructions can be used in expressions without any restrictions
(as well as all kinds of record expressions). For instance the following
expression is valid:
\begin{curry}
{person := {name := "Smith", age := 20},   -- generates a record of
 street := "Main Street",                  -- type 'Address'
 city   := "Springfield"}
\end{curry}


\subsubsection{Field Selection}
\label{sec-fieldsel}

The field selection is used to extract data from records.
It has the following form:
\begin{curry}
$r$ :> $l$
\end{curry}
It returns the value to which the label $l$ refers to from the
record expression $r$. The label must occur in the declaration of
the record type of $r$.
An example for a field selection is:
\begin{curry}
pers :> name
\end{curry}
This returns the value of the label \code{name} from the record \code{pers}
(which has the type \code{Person}).
Sequential application of field selections are also possible:
\begin{curry}
addr :> person :> age
\end{curry}
The value of the label \code{age} is extracted from a record which itself
is the value of the label \code{person} in the record \code{addr}
(which has the type \code{Address}).


\subsubsection{Field Update}
\label{sec-fieldupd}

Records can be updated by reassigning a new value to a label:
\begin{curry}
{$l_1$ := $v_1$,$\ldots$, $l_k$ := $v_k$ | $r$}
\end{curry}
The label $l_i$ is associated with the new value $v_i$ which
replaces the current value in the record $r$.
The labels must occur in the declaration
of the record type of $r$. In contrast to record constructions,
it is not necessary to specify all labels of a record.
Assignments can occur in an arbitrary order. It is not allowed to
specify more than one assignment for a label in a record update.
Examples for record updates are:
\begin{curry}
{name := "Scott", age := 25 | pers}
{person := {name := "Scott", age := 25 | pers} | addr}
\end{curry}
In these examples \code{pers} is a record of type \code{Person} and \code{addr}
is a record of type \code{Address}.


\subsubsection{Records in Pattern Matching}
\label{sec-recsinpm}

It is possible to apply pattern matching to records (e.g., in functions,
let expressions or case branches). Two kinds of record patterns
are available:
\begin{curry}
{$l_1$ = $p_1$,$\ldots$, $l_n$ = $p_n$}
{$l_1$ = $p_1$,$\ldots$, $l_k$ = $p_k$ | _}
\end{curry}
In both cases each label $l_i$ is specified with a pattern $p_i$.
All labels must occur only once in the record pattern.
The first case is used to match the whole record. Thus, all labels
of the record must occur in the pattern.
The second case is used to match only a part of
the record. Here it is not necessary to specify all labels.
This case is represented by a vertical bar followed by the underscore
(anonymous variable). It is
not allowed to use a pattern term instead of the underscore.


When trying to match a record against a record pattern, the
patterns of the specified labels are matched against
the corresponding values in the record expression. On success, all pattern
variables occurring in the patterns are replaced by their actual expression.
If none of the patterns matches, the computation fails.


Here are some examples of pattern matching with records:
\begin{curry}
isSmith30 :: Person -> Bool
isSmith30 {name = "Smith", age = 30} = True
\end{curry}
\begin{curry}
startsWith :: Char -> Person -> Bool
startsWith c {name = (d:_) | _} = c == d
\end{curry}
\begin{curry}
getPerson :: Address -> Person
getPerson {person = p | _} = p
\end{curry}
As shown in the last example, a field selection can also be obtained
by pattern matching.


\subsubsection{Export of Records}
\label{sec-exprecs}

Exporting record types and labels is very similar to exporting
data types and constructors. There are three ways
to specify an export:
\begin{itemize}
\item \code{module $M$ (\ldots, $R$, \ldots) where} \\
  exports the record $R$ without any of its labels.
\item \code{module $M$ (\ldots, $R$(..), \ldots) where} \\
  exports the record $R$ together with all its labels.
\item \code{module $M$ (\ldots, $R$($l_1$,\ldots,$l_k$), \ldots) where} \\
  exports the record $R$ together with the labels $l_1$, \ldots, $l_k$.
\end{itemize}
%
Note that imported labels cannot be overwritten in record declarations
of the importing module. It is also not possible to import equal labels
from different modules.


\subsubsection{Restrictions in the Usage of Records}
\label{sec-restrinrecs}

In contrast to the basic concept in \cite{Leijen05}, \CYS/Curry provides a
simpler version of records. Some of the features described there are
currently not supported or even restricted.

\begin{itemize}
\item Labels must be unique within the whole scope of the program.
  In particular, it is not allowed to define the same label within
  different records, not even when they are imported from other
  modules. However, it is possible to use equal identifiers for other
  entities without restrictions, since labels have an independent
  name space.
\item The record type representation with labeled fields can only be
  used as the right-hand-side of a record type declaration. It is
  not allowed to use it in any other type annotation.
\item Records are not extensible or reducible. The structure of a
  record is specified in its record declaration and cannot be
  modified at the runtime of the program.
\item Empty records are not allowed.
\item It is not allowed  to use a pattern term
  at the right side of the vertical bar in a record pattern
  except for the underscore (anonymous pattern variable).
\item Labels cannot be sequentially associated with multiple values
  (record fields do not behave like stacks).
\end{itemize}
