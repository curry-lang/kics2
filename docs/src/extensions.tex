\section{Extensions}
\label{sec-extensions}

\CYS supports some extensions in Curry programs that are not (yet)
part of the definition of Curry. These extensions are described below.

\subsection{Narrowing on Int Literals}

In addition to narrowing on algebraic data types,
\CYS also implements narrowing on values of the primitive type \code{Int}.
For example, the goal \ccode{x == 3 where x free}
is evaluated to the solutions
\begin{curry}
Prelude> x == 3 where x free
{x = (-_x2)       } False
{x = 0            } False
{x = 1            } False
{x = (2 * _x3)    } False
{x = 3            } True
{x = (4 * _x4 + 1)} False
{x = (4 * _x4 + 3)} False
\end{curry}
Note that the free variables occuring in the binding are restricted to
positive numbers greater than $0$
(the output has been indented to increase readability).
This feature is implemented by an internal binary representation
of integer numbers. If necessary, this representation can be exposed to
the user by setting the flag \code{BinaryInt} during installation:
\begin{curry}
make [kernel|install] RUNTIMEFLAGS=BinaryInt
\end{curry}
In an experimental (local) installation, the flag can also be set in the
interpreter:
\begin{curry}
:set ghc -DBinaryInt
\end{curry}
The example above will then be evaluated (without indentation) to:
\begin{curry}
Prelude> x == 3 where x free
{x = (Neg _x2)        } False
{x = 0                } False
{x = 1                } False
{x = (Pos (O _x3))    } False
{x = 3                } True
{x = (Pos (I (O _x4)))} False
{x = (Pos (I (I _x4)))} False
\end{curry}
In this output, values without free variables are presented as before.
For values containing a free variable,
the constructors \code{Neg} and \code{Pos} denote negative and
positive numbers (without $0$), while the constructors \code{O} and \code{I}
denote a $0$-- and $1$--bit where the \emph{least significant bit comes first}.
That is,
$\code{(Pos (I (O _x4)))} = +\code{(I (O _x4))} = + (2 * \code{(O _x4)}) + 1 = + (4 * \code{_x4}) + 1$
which meets the output above.

\subsection{Recursive Variable Bindings}

Local variable declarations (introduced by \code{let}\pindex{let}
or \code{where}\pindex{where}) can be (mutually) recursive in \CYS.
For instance, the declaration
\begin{curry}
ones5 = let ones = 1 : ones
         in take 5 ones
\end{curry}
introduces the local variable \code{ones} which is bound
to a \emph{cyclic structure}\index{cyclic structure}
representing an infinite list of \code{1}'s.
Similarly, the definition
\begin{curry}
onetwo n = take n one2
 where
   one2 = 1 : two1
   two1 = 2 : one2
\end{curry}
introduces a local variables \code{one2} that represents
an infinite list of alternating \code{1}'s and \code{2}'s
so that the expression \code{(onetwo 6)} evaluates to \code{[1,2,1,2,1,2]}.


\subsection{Functional Patterns}

Functional patterns \cite{AntoyHanus05LOPSTR} are a useful extension
to code operations in a more readable way. Furthermore,
defining operations with functional patterns avoids problems
caused by strict equality (\ccode{=:=}) and leads to programs
that are potentially more efficient.

Consider the definition of an operation to compute the last element
of a list \code{xs} based on the prelude operation \ccode{++}
for list concatenation:
\begin{curry}
last xs | _++[y] =:= xs  = y   where y free
\end{curry}
Since the equality constraint \ccode{=:=} evaluates both sides
to a constructor term, all elements of the list \code{xs} are
fully evaluated in order to satisfy the constraint.

Functional patterns can help to improve this computational behavior.
A \emph{functional pattern}\index{functional pattern}\index{pattern!functional}
is a function call at a pattern position. With functional patterns,
we can define the operation \code{last} as follows:
\begin{curry}
last (_++[y]) = y
\end{curry}
This definition is not only more compact but also avoids the complete
evaluation of the list elements: since a functional pattern is considered
as an abbreviation for the set of constructor terms obtained by all
evaluations of the functional pattern to normal form (see
\cite{AntoyHanus05LOPSTR} for an exact definition), the previous
definition is conceptually equivalent to the set of rules
\begin{curry}
last [y] = y
last [_,y] = y
last [_,_,y] = y
$\ldots$
\end{curry}
which shows that the evaluation of the list elements is not demanded
by the functional pattern.

In general, a pattern of the form \code{($f$ $t_1$\ldots$t_n$)} ($n>0$)
is interpreted as a functional pattern if $f$ is not a visible constructor
but a defined function that is visible in the scope of the pattern.

It is also possible to combine functional patterns with
as-patterns.\index{as-pattern}\pindex{"@}
Similarly to the meaning of as-patterns
in standard constructor patterns,
as-patterns in functional patterns are interpreted
as a sequence of pattern matching where the variable of the as-pattern
is matched before the given pattern is matched.
This process can be described by introducing an auxiliary operation
for this two-level pattern matching process.
For instance, the definition
\begin{curry}
f (_ ++ x@[(42,_)] ++ _) = x
\end{curry}
is considered as syntactic sugar for the expanded definition
\begin{curry}
f (_ ++ x ++ _) = f' x
 where
  f' [(42,_)] = x
\end{curry}
However, as-patterns are usually implemented
in a more efficient way without introducing auxiliary operations.

\subsection{Order of Pattern Matching}

Curry allows multiple occurrences of pattern variables
in standard patterns. These are an abbreviation of equational constraints
between pattern variables.
Functional patterns might also contain multiple occurrences of
pattern variables.
For instance, the operation
\begin{curry}
f (_++[x]++_++[x]++_) = x
\end{curry}
returns all elements with at least two occurrences in a list.

If functional patterns as well as multiple occurrences of
pattern variables occur in a pattern defining an operation,
there are various orders to match an expression against such
an operation. In the current implementation, the order
is as follows:
\begin{enumerate}
\item Standard pattern matching: First, it is checked whether
the constructor patterns match. Thus, functional patterns
and multiple occurrences of pattern variables are ignored.
\item Functional pattern matching: In the next phase,
functional patterns are matched but occurrences of standard
pattern variables in the functional patterns are ignored.
\item Non-linear patterns: If standard and functional pattern matching
is successful, the equational constraints which correspond
to multiple occurrences pattern variables are solved.
\item Guards: Finally, the guards supplied by the programmer
are checked.
\end{enumerate}
The order of pattern matching should not influence the computed
result. However, it might have some influence on the termination
behavior of programs, i.e., a program might not terminate
instead of finitely failing.
In such cases, it could be necessary to consider the influence
of the order of pattern matching. Note that other orders of pattern matching
can be obtained using auxiliary operations.


%%% Local Variables: 
%%% mode: latex
%%% TeX-master: "manual"
%%% End:
