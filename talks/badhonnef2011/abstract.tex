\documentclass{scrartcl}

\begin{document}

\title{KiCS2: A New Compiler from Curry to Haskell}

\author{
Bernd Bra\ss{}el
\kern1em
Michael Hanus
\kern1em
 Bj{\"o}rn Peem{\"o}ller
\kern1em
Fabian Reck \\
Institut f{\"u}r Informatik, CAU Kiel, D-24098 Kiel, Germany \\
\texttt{\{bbr|mh|bjp|fre\}@informatik.uni-kiel.de}}
\date{}
\maketitle
\thispagestyle{empty}

\begin{abstract}
We present our first steps towards a new system to compile functional logic
programs of the source language Curry into purely functional Haskell programs.
This system is based on a translation scheme where non-deterministic values
are explicitly represented, i.e., by extending each type with two additional
constructors for failed computations and for a choice between several values.
Logic variables are replaced by generator functions which preserve a
call-time-choice semantics. This enables the application of various search
strategies (e.g., depth-frist, breadth-first, parallel) to extract values from
the search space, as well as the encapsulation of non-deterministic
computations. Furthermore, deterministic computations are translated into purely
functional programs so that they are executed with almost the same efficiency as
their purely functional equivalents. Several benchmarks show that our
implementation can compete with or outperforms other existing implementations
of Curry.
\end{abstract}
\end{document}