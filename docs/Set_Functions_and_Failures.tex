\documentclass{article}

\usepackage{listings}
\lstset{basicstyle=\small\ttfamily}

\title{On the combination of Set Functions and Failures}
\author{Fabian Reck}

\begin{document}

\maketitle

\section{Towards a sound semantics}

The example that led to the need to reconsider the semantics of failure w.r.t.
Set Functions was a more complex instance of the following:

\begin{lstlisting}
data Size = Small | Big

aSize :: Size
aSize = Small ? Big

getSmallest | isEmpty (set1 smallerThan size) = size
  where size = aSize

smallerThan size | anotherSize < size = anotherSize
  where anotherSize = aSize
\end{lstlisting}

This program is a straightforward implementation of the intuitive definition
that a value from a non-deterministic expression is the smallest if there
is no smaller one. And with our implementation of Set Functions it works
just fine, there is only the one result \lstinline{Small}.

Now let's change the definition of \lstinline{getSmallest}:

\begin{lstlisting}
getSmallest 
 | isEmpty (set1 smallerThan (size =:= aSize &> size))
 = size
 where size = aSize
\end{lstlisting}

This new definition is more complex and somewhat contrived but
it still should yield the same results. The single difference is
that there is a constraint that states that there should be a
size in the results of \lstinline{sizes} that can be unified
with \lstinline{aSize}. This, of course, is trivially satisfied.

Still, with our current implementation in addition to the expected
result \lstinline{Small} there are two other ones, a second
\lstinline{Small} and \lstinline{Big}. This is obviously not intended
but can easily be explained. The new constraint introduces two
failing computations for the unification of \lstinline{Small} with
\lstinline{Big} and vice versa to the argument of 
\lstinline{isSmallest}. Since our implementation does not
capsule the non-determinism introduced by the argument and regards
a failing computation as the absence of any results, the result
of the capsuled computation is non-deterministically four sets:
$\{\texttt{Small}\},\{\},\{\}$ and $\{\}$. The first set contains
\lstinline{Small} as the result for a smaller value than \lstinline{Big},
the second one is empty because there is no smaller value than \lstinline{Small}.
The other two empty sets stem from the failing computations introduced by
the constraint.

In order to avoid the incorrect behavior we need to mark the failing computations
that are introduced by the argument to a set-valued function. In the example
above a failing computation in the argument should yield a failing computation
in the result of the set-valued function rather than an empty set of results.
This marking can be done with the same mechanism that is used to cover the
non-determinism of the argument.

The next example will show that failing computations in the argument should
not always result in a failure of the set-valued function. Consider the
following simple program:

\begin{lstlisting}
trueOrSomething x = True ? x
main = set1 trueOrSomething failed
\end{lstlisting}

Now the non-determinism is introduced not by the argument to the set-valued function
but by the function itself. Therefore, it is capsuled. That means the resulting set
should contain both \lstinline{True} and a failing computation. Since failing
computations are not considered values, there are two simple options:
\begin{enumerate}
\item A set that contains a failing computation is a failing computation
\item A set that contains failing computation is the set without the
      failing computations.
\end{enumerate}

Our first example motivates the first option. However, this option is ruled out
by the semantics of Set Functions as stated in the PPDP-Paper by Sergio Antoy
and Michael Hanus. There is specified that every result that can be obtained
from the application of the original function to some arguments has to belong
to the result of the set-valued version of the function applied to the same
arguments. In our example \lstinline{True} is a result of 
\lstinline{trueOrSomething failed}, therefore \lstinline{True} has to be an
element of a result of \lstinline{set1 trueOrSomething failed}.

The second option is a valid choice for the second example, but, as we have
already seen, it will lead to unintended results is the modified version
of the first example.

A way out of this dilemma is to combine the two options:

% TODO: Definition?
\begin{quote}
A set that contains only failing computations that stem from an argument
is a failing computation.
Otherwise, it is the set without any failing computations.
\end{quote}

This implies that the following program has to return the empty set
rather than a failing computation.

\begin{lstlisting}
failOrSomething x = failed ? x
main = set1 failOrSomething failed
\end{lstlisting}

The non-determinism is encapsulated and the resulting set contains both a failing
computation that stem from the argument and one that is introduced by the set-valued
function. 

Note that the prototypical implementation of Set Functions in PAKCS would yield no 
result for the latter two examples since the argument is strictly evaluated and
only deterministic values are passed to the set-valued function.

Following this semantics we can make an interesting observation if we consider
the following call to \lstinline{trueOrSomething}:

\begin{lstlisting}
main = set1 trueOrSomething (failed ? failed)
\end{lstlisting}

This call will yield the set $\{\texttt{True}\}$ twice. And for additional
non-deterministic occurrences of failing computations in the argument there
would be an additional result. 

So, failing computations in arguments of Set Functions can affect the
multiplicity of the results.

So far we had to distinguish failing computations in the arguments to
set-valued functions and those that were introduced by such functions.
Now we want to investigate failing computations that stem from different
levels of set-valued functions. 

We contrive an adequate example:

\begin{lstlisting}
f x = failed ? x
g x = set1 f (failed ? x)
main = set1 g failed
\end{lstlisting}

Now, what is the intended result of a call to \lstinline{main}? The non-determinism
introduced by \lstinline{f} is capsuled by the call of the set-valued version of the
function \lstinline{f} in \lstinline{g}. So the result of this call to \lstinline{f}$_{S}$
is $\{\}?\{\}$ since \lstinline{f} introduces an failure itself. The remaining non-determinism
is the one introduced by \lstinline{g}. Since \lstinline{main} calls the set-valued version
of \lstinline{g} the result of this call is $\{\{\},\{\}\}$.

Now let's change the program:

\begin{lstlisting}
f x = x
g x = set1 f (failed ? x)
main = set1 g failed
\end{lstlisting} 

Here \lstinline{f} is just an other name for \lstinline{id}. This time the
call to \lstinline{f}$_S$ by g will non-deterministically yield two
failing computations, since the non-determinism is covered an all
failing computations stem from an argument to \lstinline{f}$_S$.

In the call to \lstinline{g}$_S$ by \lstinline{main} the non-determinism is
capsuled and the resulting set contains the two failing computations. One of
these failures was introduced by \lstinline{g}, so, the result is the
empty set.

The last example regards the non-deterministic combination of failing
computations. Consider the following program:

\begin{lstlisting}
g x = set2 (?) x failed
main = set1 g failed
\end{lstlisting}

This time the non-determinism introduced by \lstinline{(?)} is capsuled and
both failures stem from arguments to \lstinline{(?)}. Therefore, the
result of \lstinline{(?)}$_S$ is a failing computation. Now we need to
determine the result of the call to \lstinline{g}$_S$ by \lstinline{main}.
The failure of the call to \lstinline{(?)}$_S$ is a result of the failure
introduced by \lstinline{g}, therefore, this call to \lstinline{g}$_S$
should yield the empty set. To be able to implement this behavior, a set
valued function that fails because of failing computations that stem
from the arguments needs to determine which of the failures stems from the
deepest nested call and has to associate this information with the resulting
failure.

\end{document}
% LocalWords:  failOrSomething trueOrSomething PPDP isSmallest aSize LocalWords
% LocalWords:  getSmallest isEmpty
