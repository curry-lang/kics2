\section{Libraries of the \CYS Distribution}
\label{sec:libraries}

{\setlength{\parindent}{0.0cm}

The \CYS distribution comes with an extensive collection
of libraries for application programming.
The libraries for meta-programming by representing
Curry programs as datatypes in Curry are described
in the following subsection in more detail.
The complete set of libraries with all exported types and functions
are described in the further subsections.
For a more detailed online documentation of all libraries of \CYS,
see \url{http://www-ps.informatik.uni-kiel.de/kics2/lib/index.html}.

\subsection{AbstractCurry and FlatCurry: Meta-Programming in Curry}
\label{sec-flatcurry}

\index{AbstractCurry}
\index{FlatCurry}
To support meta-programming, i.e., the manipulation of Curry programs
in Curry, there are system modules
\code{FlatCurry.Types} (Section~\ref{Library:FlatCurry.Types})
and \code{AbstractCurry.Types} (Section~\ref{Library:AbstractCurry.Types})
which define datatypes for the representation
of Curry programs.
\code{AbstractCurry.Types} is a more direct representation of a Curry program,
whereas \code{FlatCurry.Types} is a simplified representation
where local function definitions are replaced by global definitions
(i.e., lambda lifting has been performed) and pattern matching
is translated into explicit case/or expressions.
Thus, \code{FlatCurry.Types} can be used for more back-end oriented
program manipulations (or, for writing new back ends for Curry),
whereas \code{AbstractCurry.Types} is intended for manipulations of
programs that are more oriented towards the source program.

There are predefined I/O actions to read AbstractCurry and
FlatCurry programs: \code{AbstractCurry.Files.readCurry}\pindex{readCurry})
and \code{FlatCurry.Files.readFlatCurry}\pindex{readFlatCurry}).
These actions parse the corresponding source program and return
a data term representing this program (according to the definitions
in the modules \code{AbstractCurry.Types} and \code{FlatCurry.Types}).

Since all datatypes are explained in detail in these modules,
we refer to the online documentation\footnote{%
\url{http://www-ps.informatik.uni-kiel.de/kics2/lib/FlatCurry.Types.html} and
\url{http://www-ps.informatik.uni-kiel.de/kics2/lib/AbstractCurry.Types.html}}
of these modules.

As an example, consider a program file \ccode{test.curry}
containing the following two lines:
\begin{curry}
rev []     = []
rev (x:xs) = (rev xs) ++ [x]
\end{curry}
Then the I/O action \code{(FlatCurry.Files.readFlatCurry "test")} returns the
following term:
\begin{curry}
 (Prog "test"
  ["Prelude"]
  []
  [Func ("test","rev") 1 Public
        (FuncType (TCons ("Prelude","[]") [(TVar 0)])
                  (TCons ("Prelude","[]") [(TVar 0)]))
        (Rule [0]
           (Case Flex (Var 1)
              [Branch (Pattern ("Prelude","[]") [])
                  (Comb ConsCall ("Prelude","[]") []),
               Branch (Pattern ("Prelude",":") [2,3])
                  (Comb FuncCall ("Prelude","++")
                        [Comb FuncCall ("test","rev") [Var 3],
                         Comb ConsCall ("Prelude",":")
                              [Var 2,Comb ConsCall ("Prelude","[]") []]
                        ])
              ]))]
  []
 )
\end{curry}


%%%%%%%%%%%%%%%%%%%%%%%%%%%%%%%%%%%%%%%%%%%%%%%%%%%%%%%%%%%%%%%%%%%%%%%%%
% Definitions in order to LaTeX documents generated by "currydoc --tex"
%%%%%%%%%%%%%%%%%%%%%%%%%%%%%%%%%%%%%%%%%%%%%%%%%%%%%%%%%%%%%%%%%%%%%%%%%

\newcommand{\currymodule}[1]{\subsubsection{Library #1}\label{Library:#1}}
\newcommand{\currytypesstart}{\subsubsection*{Exported types:}}
\newcommand{\currytypesstop}{}
\newcommand{\currytypesynstart}[2]{{\tt type #2}\pindex{#1} \begin{quote}}
\newcommand{\currytypesynstop}{\end{quote}}
\newcommand{\currydatastart}[1]{{\tt data #1}\pindex{#1} \begin{quote}}
\newcommand{\currydatacons}{\end{quote}%
\begin{itemize}\item[] \hspace{-4ex}\emph{Exported constructors:}}
\newcommand{\currydatastop}{\end{itemize}}
\newcommand{\curryconsstart}[2]{\item {\tt #1~::~#2}\par}
\newcommand{\curryfuncstart}{\subsubsection*{Exported functions:}}
\newcommand{\curryfuncstop}{}
\newcommand{\curryfunctionstart}[2]{#2\pindex{#1}\begin{quote}}
\newcommand{\curryfunctionstop}{\end{quote}}
\newcommand{\curryfuncsig}[2]{{\tt #1~::~#2}}


\subsection{General Libraries}

\input{lib/AllSolutions}
\input{lib/Assertion}
\input{lib/Char}
\input{lib/Combinatorial}
\input{lib/Constraint}
\input{lib/CPNS}
\input{lib/CSV}
\input{lib/Debug}
\input{lib/Directory}
\input{lib/Distribution}
\input{lib/Either}
\input{lib/ErrorState}
\input{lib/FileGoodies}
\input{lib/FilePath}
\input{lib/Findall}
\input{lib/Float}
\input{lib/Function}
\input{lib/GetOpt}
\input{lib/Global}
\input{lib/GUI}
\input{lib/Integer}
\input{lib/IO}
\input{lib/IOExts}
\input{lib/JavaScript}
\input{lib/KeyDatabaseSQLite}
\input{lib/List}
\input{lib/Maybe}
\input{lib/NamedSocket}
\input{lib/Parser}
\input{lib/Pretty}
\input{lib/Profile}
\input{lib/Prolog}
\input{lib/PropertyFile}
\input{lib/Read}
\input{lib/ReadNumeric}
\input{lib/ReadShowTerm}
\input{lib/SetFunctions}
\input{lib/Socket}
\input{lib/System}
\input{lib/Time}
\input{lib/Unsafe}

\subsection{Data Structures and Algorithms}

\input{lib/Array}
\input{lib/Dequeue}
\input{lib/FiniteMap}
\input{lib/GraphInductive}
\input{lib/Random}
\input{lib/RedBlackTree}
\input{lib/SCC}
\input{lib/SearchTree}
\input{lib/SearchTreeTraversal}
\input{lib/SetRBT}
\input{lib/Sort}
\input{lib/TableRBT}
\input{lib/Traversal}
\input{lib/UnsafeSearchTree}
\input{lib/ValueSequence}
\input{lib/Rewriting.Term}
\input{lib/Rewriting.Substitution}
\input{lib/Rewriting.Unification}
\input{lib/Rewriting.UnificationSpec}

\subsection{Libraries for Web Applications}

\input{lib/Bootstrap3Style}
\input{lib/CategorizedHtmlList}
\input{lib/HTML}
\input{lib/HtmlCgi}
\input{lib/HtmlParser}
\input{lib/Mail}
\input{lib/Markdown}
\input{lib/URL}
\input{lib/WUI}
\input{lib/WUIjs}
\input{lib/XML}
\input{lib/XmlConv}

\subsection{Libraries for Meta-Programming}

\input{lib/AbstractCurry.Types}
\input{lib/AbstractCurry.Files}
\input{lib/AbstractCurry.Select}
\input{lib/AbstractCurry.Build}
\input{lib/AbstractCurry.Pretty}
\input{lib/FlatCurry.Types}
\input{lib/FlatCurry.Files}
\input{lib/FlatCurry.Goodies}
\input{lib/FlatCurry.Pretty}
\input{lib/FlatCurry.Read}
\input{lib/FlatCurry.Show}
\input{lib/FlatCurry.XML}
\input{lib/FlatCurry.FlexRigid}
\input{lib/FlatCurry.Compact}
\input{lib/FlatCurry.Annotated.Types}
\input{lib/FlatCurry.Annotated.Pretty}
\input{lib/FlatCurry.Annotated.Goodies}
\input{lib/FlatCurry.Annotated.TypeSubst}
\input{lib/FlatCurry.Annotated.TypeInference}
\input{lib/CurryStringClassifier}

} % end setlength parindent


%%% Local Variables: 
%%% mode: latex
%%% TeX-master: "manual"
%%% End: 
